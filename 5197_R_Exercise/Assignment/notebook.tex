
% Default to the notebook output style

    


% Inherit from the specified cell style.




    
\documentclass[11pt]{article}

    
    
    \usepackage[T1]{fontenc}
    % Nicer default font (+ math font) than Computer Modern for most use cases
    \usepackage{mathpazo}

    % Basic figure setup, for now with no caption control since it's done
    % automatically by Pandoc (which extracts ![](path) syntax from Markdown).
    \usepackage{graphicx}
    % We will generate all images so they have a width \maxwidth. This means
    % that they will get their normal width if they fit onto the page, but
    % are scaled down if they would overflow the margins.
    \makeatletter
    \def\maxwidth{\ifdim\Gin@nat@width>\linewidth\linewidth
    \else\Gin@nat@width\fi}
    \makeatother
    \let\Oldincludegraphics\includegraphics
    % Set max figure width to be 80% of text width, for now hardcoded.
    \renewcommand{\includegraphics}[1]{\Oldincludegraphics[width=.8\maxwidth]{#1}}
    % Ensure that by default, figures have no caption (until we provide a
    % proper Figure object with a Caption API and a way to capture that
    % in the conversion process - todo).
    \usepackage{caption}
    \DeclareCaptionLabelFormat{nolabel}{}
    \captionsetup{labelformat=nolabel}

    \usepackage{adjustbox} % Used to constrain images to a maximum size 
    \usepackage{xcolor} % Allow colors to be defined
    \usepackage{enumerate} % Needed for markdown enumerations to work
    \usepackage{geometry} % Used to adjust the document margins
    \usepackage{amsmath} % Equations
    \usepackage{amssymb} % Equations
    \usepackage{textcomp} % defines textquotesingle
    % Hack from http://tex.stackexchange.com/a/47451/13684:
    \AtBeginDocument{%
        \def\PYZsq{\textquotesingle}% Upright quotes in Pygmentized code
    }
    \usepackage{upquote} % Upright quotes for verbatim code
    \usepackage{eurosym} % defines \euro
    \usepackage[mathletters]{ucs} % Extended unicode (utf-8) support
    \usepackage[utf8x]{inputenc} % Allow utf-8 characters in the tex document
    \usepackage{fancyvrb} % verbatim replacement that allows latex
    \usepackage{grffile} % extends the file name processing of package graphics 
                         % to support a larger range 
    % The hyperref package gives us a pdf with properly built
    % internal navigation ('pdf bookmarks' for the table of contents,
    % internal cross-reference links, web links for URLs, etc.)
    \usepackage{hyperref}
    \usepackage{longtable} % longtable support required by pandoc >1.10
    \usepackage{booktabs}  % table support for pandoc > 1.12.2
    \usepackage[inline]{enumitem} % IRkernel/repr support (it uses the enumerate* environment)
    \usepackage[normalem]{ulem} % ulem is needed to support strikethroughs (\sout)
                                % normalem makes italics be italics, not underlines
    

    
    
    % Colors for the hyperref package
    \definecolor{urlcolor}{rgb}{0,.145,.698}
    \definecolor{linkcolor}{rgb}{.71,0.21,0.01}
    \definecolor{citecolor}{rgb}{.12,.54,.11}

    % ANSI colors
    \definecolor{ansi-black}{HTML}{3E424D}
    \definecolor{ansi-black-intense}{HTML}{282C36}
    \definecolor{ansi-red}{HTML}{E75C58}
    \definecolor{ansi-red-intense}{HTML}{B22B31}
    \definecolor{ansi-green}{HTML}{00A250}
    \definecolor{ansi-green-intense}{HTML}{007427}
    \definecolor{ansi-yellow}{HTML}{DDB62B}
    \definecolor{ansi-yellow-intense}{HTML}{B27D12}
    \definecolor{ansi-blue}{HTML}{208FFB}
    \definecolor{ansi-blue-intense}{HTML}{0065CA}
    \definecolor{ansi-magenta}{HTML}{D160C4}
    \definecolor{ansi-magenta-intense}{HTML}{A03196}
    \definecolor{ansi-cyan}{HTML}{60C6C8}
    \definecolor{ansi-cyan-intense}{HTML}{258F8F}
    \definecolor{ansi-white}{HTML}{C5C1B4}
    \definecolor{ansi-white-intense}{HTML}{A1A6B2}

    % commands and environments needed by pandoc snippets
    % extracted from the output of `pandoc -s`
    \providecommand{\tightlist}{%
      \setlength{\itemsep}{0pt}\setlength{\parskip}{0pt}}
    \DefineVerbatimEnvironment{Highlighting}{Verbatim}{commandchars=\\\{\}}
    % Add ',fontsize=\small' for more characters per line
    \newenvironment{Shaded}{}{}
    \newcommand{\KeywordTok}[1]{\textcolor[rgb]{0.00,0.44,0.13}{\textbf{{#1}}}}
    \newcommand{\DataTypeTok}[1]{\textcolor[rgb]{0.56,0.13,0.00}{{#1}}}
    \newcommand{\DecValTok}[1]{\textcolor[rgb]{0.25,0.63,0.44}{{#1}}}
    \newcommand{\BaseNTok}[1]{\textcolor[rgb]{0.25,0.63,0.44}{{#1}}}
    \newcommand{\FloatTok}[1]{\textcolor[rgb]{0.25,0.63,0.44}{{#1}}}
    \newcommand{\CharTok}[1]{\textcolor[rgb]{0.25,0.44,0.63}{{#1}}}
    \newcommand{\StringTok}[1]{\textcolor[rgb]{0.25,0.44,0.63}{{#1}}}
    \newcommand{\CommentTok}[1]{\textcolor[rgb]{0.38,0.63,0.69}{\textit{{#1}}}}
    \newcommand{\OtherTok}[1]{\textcolor[rgb]{0.00,0.44,0.13}{{#1}}}
    \newcommand{\AlertTok}[1]{\textcolor[rgb]{1.00,0.00,0.00}{\textbf{{#1}}}}
    \newcommand{\FunctionTok}[1]{\textcolor[rgb]{0.02,0.16,0.49}{{#1}}}
    \newcommand{\RegionMarkerTok}[1]{{#1}}
    \newcommand{\ErrorTok}[1]{\textcolor[rgb]{1.00,0.00,0.00}{\textbf{{#1}}}}
    \newcommand{\NormalTok}[1]{{#1}}
    
    % Additional commands for more recent versions of Pandoc
    \newcommand{\ConstantTok}[1]{\textcolor[rgb]{0.53,0.00,0.00}{{#1}}}
    \newcommand{\SpecialCharTok}[1]{\textcolor[rgb]{0.25,0.44,0.63}{{#1}}}
    \newcommand{\VerbatimStringTok}[1]{\textcolor[rgb]{0.25,0.44,0.63}{{#1}}}
    \newcommand{\SpecialStringTok}[1]{\textcolor[rgb]{0.73,0.40,0.53}{{#1}}}
    \newcommand{\ImportTok}[1]{{#1}}
    \newcommand{\DocumentationTok}[1]{\textcolor[rgb]{0.73,0.13,0.13}{\textit{{#1}}}}
    \newcommand{\AnnotationTok}[1]{\textcolor[rgb]{0.38,0.63,0.69}{\textbf{\textit{{#1}}}}}
    \newcommand{\CommentVarTok}[1]{\textcolor[rgb]{0.38,0.63,0.69}{\textbf{\textit{{#1}}}}}
    \newcommand{\VariableTok}[1]{\textcolor[rgb]{0.10,0.09,0.49}{{#1}}}
    \newcommand{\ControlFlowTok}[1]{\textcolor[rgb]{0.00,0.44,0.13}{\textbf{{#1}}}}
    \newcommand{\OperatorTok}[1]{\textcolor[rgb]{0.40,0.40,0.40}{{#1}}}
    \newcommand{\BuiltInTok}[1]{{#1}}
    \newcommand{\ExtensionTok}[1]{{#1}}
    \newcommand{\PreprocessorTok}[1]{\textcolor[rgb]{0.74,0.48,0.00}{{#1}}}
    \newcommand{\AttributeTok}[1]{\textcolor[rgb]{0.49,0.56,0.16}{{#1}}}
    \newcommand{\InformationTok}[1]{\textcolor[rgb]{0.38,0.63,0.69}{\textbf{\textit{{#1}}}}}
    \newcommand{\WarningTok}[1]{\textcolor[rgb]{0.38,0.63,0.69}{\textbf{\textit{{#1}}}}}
    
    
    % Define a nice break command that doesn't care if a line doesn't already
    % exist.
    \def\br{\hspace*{\fill} \\* }
    % Math Jax compatability definitions
    \def\gt{>}
    \def\lt{<}
    % Document parameters
    \title{27771539\_ass1\_FIT5197}
    
    
    

    % Pygments definitions
    
\makeatletter
\def\PY@reset{\let\PY@it=\relax \let\PY@bf=\relax%
    \let\PY@ul=\relax \let\PY@tc=\relax%
    \let\PY@bc=\relax \let\PY@ff=\relax}
\def\PY@tok#1{\csname PY@tok@#1\endcsname}
\def\PY@toks#1+{\ifx\relax#1\empty\else%
    \PY@tok{#1}\expandafter\PY@toks\fi}
\def\PY@do#1{\PY@bc{\PY@tc{\PY@ul{%
    \PY@it{\PY@bf{\PY@ff{#1}}}}}}}
\def\PY#1#2{\PY@reset\PY@toks#1+\relax+\PY@do{#2}}

\expandafter\def\csname PY@tok@w\endcsname{\def\PY@tc##1{\textcolor[rgb]{0.73,0.73,0.73}{##1}}}
\expandafter\def\csname PY@tok@c\endcsname{\let\PY@it=\textit\def\PY@tc##1{\textcolor[rgb]{0.25,0.50,0.50}{##1}}}
\expandafter\def\csname PY@tok@cp\endcsname{\def\PY@tc##1{\textcolor[rgb]{0.74,0.48,0.00}{##1}}}
\expandafter\def\csname PY@tok@k\endcsname{\let\PY@bf=\textbf\def\PY@tc##1{\textcolor[rgb]{0.00,0.50,0.00}{##1}}}
\expandafter\def\csname PY@tok@kp\endcsname{\def\PY@tc##1{\textcolor[rgb]{0.00,0.50,0.00}{##1}}}
\expandafter\def\csname PY@tok@kt\endcsname{\def\PY@tc##1{\textcolor[rgb]{0.69,0.00,0.25}{##1}}}
\expandafter\def\csname PY@tok@o\endcsname{\def\PY@tc##1{\textcolor[rgb]{0.40,0.40,0.40}{##1}}}
\expandafter\def\csname PY@tok@ow\endcsname{\let\PY@bf=\textbf\def\PY@tc##1{\textcolor[rgb]{0.67,0.13,1.00}{##1}}}
\expandafter\def\csname PY@tok@nb\endcsname{\def\PY@tc##1{\textcolor[rgb]{0.00,0.50,0.00}{##1}}}
\expandafter\def\csname PY@tok@nf\endcsname{\def\PY@tc##1{\textcolor[rgb]{0.00,0.00,1.00}{##1}}}
\expandafter\def\csname PY@tok@nc\endcsname{\let\PY@bf=\textbf\def\PY@tc##1{\textcolor[rgb]{0.00,0.00,1.00}{##1}}}
\expandafter\def\csname PY@tok@nn\endcsname{\let\PY@bf=\textbf\def\PY@tc##1{\textcolor[rgb]{0.00,0.00,1.00}{##1}}}
\expandafter\def\csname PY@tok@ne\endcsname{\let\PY@bf=\textbf\def\PY@tc##1{\textcolor[rgb]{0.82,0.25,0.23}{##1}}}
\expandafter\def\csname PY@tok@nv\endcsname{\def\PY@tc##1{\textcolor[rgb]{0.10,0.09,0.49}{##1}}}
\expandafter\def\csname PY@tok@no\endcsname{\def\PY@tc##1{\textcolor[rgb]{0.53,0.00,0.00}{##1}}}
\expandafter\def\csname PY@tok@nl\endcsname{\def\PY@tc##1{\textcolor[rgb]{0.63,0.63,0.00}{##1}}}
\expandafter\def\csname PY@tok@ni\endcsname{\let\PY@bf=\textbf\def\PY@tc##1{\textcolor[rgb]{0.60,0.60,0.60}{##1}}}
\expandafter\def\csname PY@tok@na\endcsname{\def\PY@tc##1{\textcolor[rgb]{0.49,0.56,0.16}{##1}}}
\expandafter\def\csname PY@tok@nt\endcsname{\let\PY@bf=\textbf\def\PY@tc##1{\textcolor[rgb]{0.00,0.50,0.00}{##1}}}
\expandafter\def\csname PY@tok@nd\endcsname{\def\PY@tc##1{\textcolor[rgb]{0.67,0.13,1.00}{##1}}}
\expandafter\def\csname PY@tok@s\endcsname{\def\PY@tc##1{\textcolor[rgb]{0.73,0.13,0.13}{##1}}}
\expandafter\def\csname PY@tok@sd\endcsname{\let\PY@it=\textit\def\PY@tc##1{\textcolor[rgb]{0.73,0.13,0.13}{##1}}}
\expandafter\def\csname PY@tok@si\endcsname{\let\PY@bf=\textbf\def\PY@tc##1{\textcolor[rgb]{0.73,0.40,0.53}{##1}}}
\expandafter\def\csname PY@tok@se\endcsname{\let\PY@bf=\textbf\def\PY@tc##1{\textcolor[rgb]{0.73,0.40,0.13}{##1}}}
\expandafter\def\csname PY@tok@sr\endcsname{\def\PY@tc##1{\textcolor[rgb]{0.73,0.40,0.53}{##1}}}
\expandafter\def\csname PY@tok@ss\endcsname{\def\PY@tc##1{\textcolor[rgb]{0.10,0.09,0.49}{##1}}}
\expandafter\def\csname PY@tok@sx\endcsname{\def\PY@tc##1{\textcolor[rgb]{0.00,0.50,0.00}{##1}}}
\expandafter\def\csname PY@tok@m\endcsname{\def\PY@tc##1{\textcolor[rgb]{0.40,0.40,0.40}{##1}}}
\expandafter\def\csname PY@tok@gh\endcsname{\let\PY@bf=\textbf\def\PY@tc##1{\textcolor[rgb]{0.00,0.00,0.50}{##1}}}
\expandafter\def\csname PY@tok@gu\endcsname{\let\PY@bf=\textbf\def\PY@tc##1{\textcolor[rgb]{0.50,0.00,0.50}{##1}}}
\expandafter\def\csname PY@tok@gd\endcsname{\def\PY@tc##1{\textcolor[rgb]{0.63,0.00,0.00}{##1}}}
\expandafter\def\csname PY@tok@gi\endcsname{\def\PY@tc##1{\textcolor[rgb]{0.00,0.63,0.00}{##1}}}
\expandafter\def\csname PY@tok@gr\endcsname{\def\PY@tc##1{\textcolor[rgb]{1.00,0.00,0.00}{##1}}}
\expandafter\def\csname PY@tok@ge\endcsname{\let\PY@it=\textit}
\expandafter\def\csname PY@tok@gs\endcsname{\let\PY@bf=\textbf}
\expandafter\def\csname PY@tok@gp\endcsname{\let\PY@bf=\textbf\def\PY@tc##1{\textcolor[rgb]{0.00,0.00,0.50}{##1}}}
\expandafter\def\csname PY@tok@go\endcsname{\def\PY@tc##1{\textcolor[rgb]{0.53,0.53,0.53}{##1}}}
\expandafter\def\csname PY@tok@gt\endcsname{\def\PY@tc##1{\textcolor[rgb]{0.00,0.27,0.87}{##1}}}
\expandafter\def\csname PY@tok@err\endcsname{\def\PY@bc##1{\setlength{\fboxsep}{0pt}\fcolorbox[rgb]{1.00,0.00,0.00}{1,1,1}{\strut ##1}}}
\expandafter\def\csname PY@tok@kc\endcsname{\let\PY@bf=\textbf\def\PY@tc##1{\textcolor[rgb]{0.00,0.50,0.00}{##1}}}
\expandafter\def\csname PY@tok@kd\endcsname{\let\PY@bf=\textbf\def\PY@tc##1{\textcolor[rgb]{0.00,0.50,0.00}{##1}}}
\expandafter\def\csname PY@tok@kn\endcsname{\let\PY@bf=\textbf\def\PY@tc##1{\textcolor[rgb]{0.00,0.50,0.00}{##1}}}
\expandafter\def\csname PY@tok@kr\endcsname{\let\PY@bf=\textbf\def\PY@tc##1{\textcolor[rgb]{0.00,0.50,0.00}{##1}}}
\expandafter\def\csname PY@tok@bp\endcsname{\def\PY@tc##1{\textcolor[rgb]{0.00,0.50,0.00}{##1}}}
\expandafter\def\csname PY@tok@fm\endcsname{\def\PY@tc##1{\textcolor[rgb]{0.00,0.00,1.00}{##1}}}
\expandafter\def\csname PY@tok@vc\endcsname{\def\PY@tc##1{\textcolor[rgb]{0.10,0.09,0.49}{##1}}}
\expandafter\def\csname PY@tok@vg\endcsname{\def\PY@tc##1{\textcolor[rgb]{0.10,0.09,0.49}{##1}}}
\expandafter\def\csname PY@tok@vi\endcsname{\def\PY@tc##1{\textcolor[rgb]{0.10,0.09,0.49}{##1}}}
\expandafter\def\csname PY@tok@vm\endcsname{\def\PY@tc##1{\textcolor[rgb]{0.10,0.09,0.49}{##1}}}
\expandafter\def\csname PY@tok@sa\endcsname{\def\PY@tc##1{\textcolor[rgb]{0.73,0.13,0.13}{##1}}}
\expandafter\def\csname PY@tok@sb\endcsname{\def\PY@tc##1{\textcolor[rgb]{0.73,0.13,0.13}{##1}}}
\expandafter\def\csname PY@tok@sc\endcsname{\def\PY@tc##1{\textcolor[rgb]{0.73,0.13,0.13}{##1}}}
\expandafter\def\csname PY@tok@dl\endcsname{\def\PY@tc##1{\textcolor[rgb]{0.73,0.13,0.13}{##1}}}
\expandafter\def\csname PY@tok@s2\endcsname{\def\PY@tc##1{\textcolor[rgb]{0.73,0.13,0.13}{##1}}}
\expandafter\def\csname PY@tok@sh\endcsname{\def\PY@tc##1{\textcolor[rgb]{0.73,0.13,0.13}{##1}}}
\expandafter\def\csname PY@tok@s1\endcsname{\def\PY@tc##1{\textcolor[rgb]{0.73,0.13,0.13}{##1}}}
\expandafter\def\csname PY@tok@mb\endcsname{\def\PY@tc##1{\textcolor[rgb]{0.40,0.40,0.40}{##1}}}
\expandafter\def\csname PY@tok@mf\endcsname{\def\PY@tc##1{\textcolor[rgb]{0.40,0.40,0.40}{##1}}}
\expandafter\def\csname PY@tok@mh\endcsname{\def\PY@tc##1{\textcolor[rgb]{0.40,0.40,0.40}{##1}}}
\expandafter\def\csname PY@tok@mi\endcsname{\def\PY@tc##1{\textcolor[rgb]{0.40,0.40,0.40}{##1}}}
\expandafter\def\csname PY@tok@il\endcsname{\def\PY@tc##1{\textcolor[rgb]{0.40,0.40,0.40}{##1}}}
\expandafter\def\csname PY@tok@mo\endcsname{\def\PY@tc##1{\textcolor[rgb]{0.40,0.40,0.40}{##1}}}
\expandafter\def\csname PY@tok@ch\endcsname{\let\PY@it=\textit\def\PY@tc##1{\textcolor[rgb]{0.25,0.50,0.50}{##1}}}
\expandafter\def\csname PY@tok@cm\endcsname{\let\PY@it=\textit\def\PY@tc##1{\textcolor[rgb]{0.25,0.50,0.50}{##1}}}
\expandafter\def\csname PY@tok@cpf\endcsname{\let\PY@it=\textit\def\PY@tc##1{\textcolor[rgb]{0.25,0.50,0.50}{##1}}}
\expandafter\def\csname PY@tok@c1\endcsname{\let\PY@it=\textit\def\PY@tc##1{\textcolor[rgb]{0.25,0.50,0.50}{##1}}}
\expandafter\def\csname PY@tok@cs\endcsname{\let\PY@it=\textit\def\PY@tc##1{\textcolor[rgb]{0.25,0.50,0.50}{##1}}}

\def\PYZbs{\char`\\}
\def\PYZus{\char`\_}
\def\PYZob{\char`\{}
\def\PYZcb{\char`\}}
\def\PYZca{\char`\^}
\def\PYZam{\char`\&}
\def\PYZlt{\char`\<}
\def\PYZgt{\char`\>}
\def\PYZsh{\char`\#}
\def\PYZpc{\char`\%}
\def\PYZdl{\char`\$}
\def\PYZhy{\char`\-}
\def\PYZsq{\char`\'}
\def\PYZdq{\char`\"}
\def\PYZti{\char`\~}
% for compatibility with earlier versions
\def\PYZat{@}
\def\PYZlb{[}
\def\PYZrb{]}
\makeatother


    % Exact colors from NB
    \definecolor{incolor}{rgb}{0.0, 0.0, 0.5}
    \definecolor{outcolor}{rgb}{0.545, 0.0, 0.0}



    
    % Prevent overflowing lines due to hard-to-break entities
    \sloppy 
    % Setup hyperref package
    \hypersetup{
      breaklinks=true,  % so long urls are correctly broken across lines
      colorlinks=true,
      urlcolor=urlcolor,
      linkcolor=linkcolor,
      citecolor=citecolor,
      }
    % Slightly bigger margins than the latex defaults
    
    \geometry{verbose,tmargin=1in,bmargin=1in,lmargin=1in,rmargin=1in}
    
    

    \begin{document}
    
    
    \maketitle
    
    

    
    \textbf{Title}: FIT5197 2018 S1 Assignment 1 \textbf{Author}: Chuangfu
Xie, 27771539\\
\textbf{Date}: 13 April 2018

    \subsection{Question 1: Calculate conditional probability of an
event}\label{question-1-calculate-conditional-probability-of-an-event}

\textbf{Solution}:\\
Let \textbf{\emph{A}} denote the event that each value appears at least
once, and \textbf{\emph{B}} the event that the outcome is alternate in
numbers. To compute \textbf{\emph{P(A\textbar{}B)}}, we have\\
\[P(A|B)=\frac { P(AB) }{ P(B) }\] Firstly, we compute
\textbf{\emph{P(AB)}}, to have the intersection of event
\textbf{\emph{A}} and \textbf{\emph{B}}:\\
1. In general, for a 7 time tossing, each tossing will yield a value
contributing to the outcome. Since a dice only have 6 possible values,
we have \[{ 6 }^{ 7 }=279936\] 1. As event \textbf{\emph{A}} state that
each value appears at least once, that is, to have a combination of 6
value without duplication, we have \[{ A }_{ 6 }^{ 6 }=6!=720\] 2. Then,
we consider event \textbf{\emph{B}} which state that there are no two
values are adjacent. To meet this condition, as aforementioned
permutation have 6 unique values, what we do is selecting a value from 1
to 6 and inserting it into 7 "interval" between these 6 values. we have
\[(5+4+4+4+4+4+5)\div2=15\]

\begin{quote}
For example, to insert a value right before the 1st value, the possible
value have to be in the 5 values of which have excluded the value of
following one, that is, 5 option for us to choose (Similar to inserting
a value behind the 6th value). For the "interval" where both side have
different values, on the analogy of previous reasoning, only 4 values
are available for us to pick. As in the combination, there are 7
"intervals", only 5 "interval" are for this case.

\begin{longtable}[]{@{}llcccccr@{}}
\toprule
"Intervals" & \({1}_{st}\) & \({2}_{nd}\) & \({3}_{rd}\) & \({4}_{th}\)
& \({5}_{th}\) & \({6}_{th}\) & \({7}_{th}\)\tabularnewline
\midrule
\endhead
options & 5 & 4 & 4 & 4 & 4 & 4 & 5\tabularnewline
\bottomrule
\end{longtable}

However, let's take a general situation as example: For 6 value
combination, says (1,2,3,4,5,6), inserting a value, which is identical
to the last value, into any "interval" from 1st to 5th, for instance,
(6,1,2,3,4,5,6), is exactly the same as the situation where inserting a
value right behind the tail of a given combination say (6,1,2,3,4,5),
which becomes (6,1,2,3,4,5,6) in the end.\\
Hence, the options we calculated so far have to be halved so that the
total possible options for us is 15.
\end{quote}

To sum up, we have
\[P(AB)=\frac { 15\times { A }_{ 6 }^{ 6 } }{ { 6 }^{ 7 } } =\frac { 15\times 720 }{ 279936 } =\frac { 25 }{ 648 }\approx 0.03858 \]

Next, we are going to compute \textbf{\emph{P(B)}}:\\
According to previous reasoning, if event B occurs, that is, no two
value adjacent having the same value. Hence, for each value in each
position, we denote each value as \({ B }_{ i }\) where i are the
position of the value. As each value are independent to others, we have
\[B={ B }_{ 1 }{ B }_{ 2 }{ B }_{ 3 }{ B }_{ 4 }{ B }_{ 5 }{ B }_{ 6 }{ B }_{ 7 }\]
then, we have
\[P(B)=P({ B }_{ 1 }{ B }_{ 2 }{ B }_{ 3 }{ B }_{ 4 }{ B }_{ 5 }{ B }_{ 6 }{ B }_{ 7 })\]
\[\Rightarrow P(B)=P({ B }_{ 1 }{ )\cdot P(B }_{ 2 }|{ B }_{ 1 }){ \cdot P(B }_{ 3 }|{ B }_{ 1 }{ B }_{ 2 }){ \cdot P(B }_{ 4 }|{ B }_{ 1 }{ B }_{ 2 }{ B }_{ 3 })\\ \quad \quad \quad \quad \quad { \cdot P(B }_{ 5 }|{ B }_{ 1 }{ B }_{ 2 }{ B }_{ 3 }{ B }_{ 4 }){ \cdot P(B }_{ 6 }|{ B }_{ 1 }{ B }_{ 2 }{ B }_{ 3 }{ B }_{ 4 }{ B }_{ 5 })\\ \quad \quad \quad \quad \quad { \cdot P(B }_{ 7 }|{ B }_{ 1 }{ B }_{ 2 }{ B }_{ 3 }{ B }_{ 4 }{ B }_{ 5 }{ B }_{ 6 })\]
Since we have \[P({ B }_{ 1 })=1\]
\[P({ B }_{ 2 }|{ B }_{ 1 })=\frac { P({ B }_{ 2 }{ B }_{ 1 }) }{ P({ B }_{ 1 })}=\cfrac { 1\times \cfrac { 5 }{ 6 }  }{ 1 } =\cfrac { 5 }{ 6 } \]
\[P({ { B }_{ 3 }|B }_{ 2 }{ B }_{ 1 })=\frac { P({ { B }_{ 3 }B }_{ 2 }{ B }_{ 1 }) }{ P({ { B }_{ 2 }B }_{ 1 }) } =\cfrac { \cfrac { 5 }{ 6 } \times \cfrac { 5 }{ 6 }  }{ \cfrac { 5 }{ 6 }  } =\cfrac { 5 }{ 6 } \]
\[\cdots \]
\[{ P(B }_{ 7 }|{ B }_{ 1 }{ B }_{ 2 }{ B }_{ 3 }{ B }_{ 4 }{ B }_{ 5 }{ B }_{ 6 })=\frac { P({ B }_{ 7 }{ { B }_{ 6 }{ B }_{ 5 } }{ B }_{ 4 }{ B }_{ 3 }{ B }_{ 2 }{ B }_{ 1 }) }{ P({ { B }_{ 6 }{ B }_{ 5 } }{ B }_{ 4 }{ B }_{ 3 }{ B }_{ 2 }{ B }_{ 1 }) } =\cfrac { \cfrac { 5 }{ 6 } \times \cfrac { 5 }{ 6 }  }{ \cfrac { 5 }{ 6 }  } =\cfrac { 5 }{ 6 } \]
To sum up
\[{ P(B })=1\times \cfrac { 5 }{ 6 } \times \cfrac { 5 }{ 6 } \times \cfrac { 5 }{ 6 } \times \cfrac { 5 }{ 6 } \times \cfrac { 5 }{ 6 } \times \cfrac { 5 }{ 6 } \approx 0.3349\]

Finally, we can compute the \textbf{\emph{P(A\textbar{}B)}}:
\[P(A|B)=\frac { P(AB) }{ P(B) }=\cfrac { \frac { 25 }{ 648 }  }{ 1\times \cfrac { 5 }{ 6 } \times \cfrac { 5 }{ 6 } \times \cfrac { 5 }{ 6 } \times \cfrac { 5 }{ 6 } \times \cfrac { 5 }{ 6 } \times \cfrac { 5 }{ 6 }  } = 0.1152\]

    \begin{Verbatim}[commandchars=\\\{\}]
{\color{incolor}In [{\color{incolor}67}]:} nRuns \PY{o}{=} \PY{l+m}{100000}
         AB \PY{o}{=} \PY{l+m}{0} 
         B \PY{o}{=} \PY{l+m}{0}
         \PY{k+kr}{for} \PY{p}{(}i \PY{k+kr}{in} \PY{l+m}{1}\PY{o}{:}nRuns\PY{p}{)}\PY{p}{\PYZob{}}
             dice.roll \PY{o}{\PYZlt{}\PYZhy{}} \PY{k+kp}{sample}\PY{p}{(}\PY{k+kt}{c}\PY{p}{(}\PY{l+m}{1}\PY{o}{:}\PY{l+m}{6}\PY{p}{)}\PY{p}{,} \PY{l+m}{7}\PY{p}{,} replace \PY{o}{=} \PY{n+nb+bp}{T} \PY{p}{)}
             A.result \PY{o}{\PYZlt{}\PYZhy{}} \PY{o}{!}dice.roll \PY{o}{\PYZpc{}in\PYZpc{}} dice.roll\PY{p}{[}\PY{k+kp}{duplicated}\PY{p}{(}dice.roll\PY{p}{)}\PY{p}{]}
             \PY{c+c1}{\PYZsh{} For computing p(AB)}
             \PY{k+kr}{if}\PY{p}{(}\PY{k+kp}{sum}\PY{p}{(}A.result\PY{p}{)} \PY{o}{==} \PY{l+m}{5}\PY{p}{)}\PY{p}{\PYZob{}}
                 \PY{c+c1}{\PYZsh{}print(dice.roll)}
                 \PY{c+c1}{\PYZsh{}A = A + 1}
                 AB.pre \PY{o}{=} \PY{l+m}{0}
                 AB.cnt \PY{o}{=} \PY{l+m}{0}
                 \PY{k+kr}{for}\PY{p}{(}i \PY{k+kr}{in} dice.roll\PY{p}{)}\PY{p}{\PYZob{}}
                     \PY{k+kr}{if} \PY{p}{(}AB.pre \PY{o}{==}\PY{l+m}{0}\PY{p}{)}\PY{p}{\PYZob{}}
                         AB.pre \PY{o}{\PYZlt{}\PYZhy{}} i \PY{c+c1}{\PYZsh{}ini}
                     \PY{p}{\PYZcb{}}\PY{k+kr}{else} \PY{k+kr}{if}\PY{p}{(}\PY{k+kp}{abs}\PY{p}{(}pre\PY{o}{\PYZhy{}}i\PY{p}{)} \PY{o}{!=} \PY{l+m}{0}\PY{p}{)}\PY{p}{\PYZob{}}
                         AB.cnt \PY{o}{=} AB.cnt \PY{o}{+} \PY{l+m}{1}
                     \PY{p}{\PYZcb{}}
                     pre \PY{o}{\PYZlt{}\PYZhy{}} i
                 \PY{p}{\PYZcb{}}
                 \PY{k+kr}{if}\PY{p}{(}AB.cnt \PY{o}{==} \PY{l+m}{6}\PY{p}{)}\PY{p}{\PYZob{}}
                     \PY{c+c1}{\PYZsh{}print(dice.roll)}
                     AB \PY{o}{=} AB \PY{o}{+} \PY{l+m}{1}
                 \PY{p}{\PYZcb{}}
             \PY{p}{\PYZcb{}}
             \PY{c+c1}{\PYZsh{} For computing P(B)}
             B.pre \PY{o}{=} \PY{l+m}{0} 
             B.cnt \PY{o}{=} \PY{l+m}{0}
             \PY{k+kr}{for}\PY{p}{(}i \PY{k+kr}{in} dice.roll\PY{p}{)}\PY{p}{\PYZob{}}
                 \PY{k+kr}{if} \PY{p}{(}B.pre \PY{o}{==}\PY{l+m}{0}\PY{p}{)}\PY{p}{\PYZob{}}
                     B.pre \PY{o}{\PYZlt{}\PYZhy{}} i \PY{c+c1}{\PYZsh{}ini}
                 \PY{p}{\PYZcb{}}\PY{k+kr}{else} \PY{k+kr}{if}\PY{p}{(}\PY{k+kp}{abs}\PY{p}{(}pre\PY{o}{\PYZhy{}}i\PY{p}{)} \PY{o}{!=} \PY{l+m}{0}\PY{p}{)}\PY{p}{\PYZob{}}
                     B.cnt \PY{o}{=} B.cnt \PY{o}{+} \PY{l+m}{1}
                 \PY{p}{\PYZcb{}}
                 pre \PY{o}{\PYZlt{}\PYZhy{}} i
             \PY{p}{\PYZcb{}}
             \PY{k+kr}{if}\PY{p}{(}B.cnt \PY{o}{==} \PY{l+m}{6}\PY{p}{)}\PY{p}{\PYZob{}}
                 \PY{c+c1}{\PYZsh{}print(dice.roll)}
                 B \PY{o}{=} B \PY{o}{+} \PY{l+m}{1}
             \PY{p}{\PYZcb{}}
         \PY{p}{\PYZcb{}}
         \PY{k+kp}{print}\PY{p}{(}\PY{l+s}{\PYZdq{}}\PY{l+s}{Simulation result:\PYZdq{}}\PY{p}{)}
         \PY{k+kp}{print}\PY{p}{(}\PY{k+kp}{paste}\PY{p}{(}\PY{l+s}{\PYZdq{}}\PY{l+s}{P(AB)=\PYZdq{}}\PY{p}{,}AB\PY{o}{/}nRuns\PY{p}{)}\PY{p}{)}
         \PY{k+kp}{print}\PY{p}{(}\PY{k+kp}{paste}\PY{p}{(}\PY{l+s}{\PYZdq{}}\PY{l+s}{P(B)=\PYZdq{}}\PY{p}{,}B\PY{o}{/}nRuns\PY{p}{)}\PY{p}{)}
         \PY{k+kp}{print}\PY{p}{(}\PY{k+kp}{paste}\PY{p}{(}\PY{l+s}{\PYZdq{}}\PY{l+s}{P(A|B)=\PYZdq{}}\PY{p}{,}AB\PY{o}{/}B\PY{p}{)}\PY{p}{)}
\end{Verbatim}


    \begin{Verbatim}[commandchars=\\\{\}]
[1] "Simulation result:"
[1] "P(AB)= 0.03854"
[1] "P(B)= 0.33499"
[1] "P(A|B)= 0.11504821039434"

    \end{Verbatim}

    \subsection{Question 2: Entropy}\label{question-2-entropy}

    \textbf{Solution}:\\
First, read data from folder:\\
\textbf{Note}: To successfully read data from disk must have this
notebook and the csv file "FIT5197\_2018\_S1\_Assignment1\_Q2\_data.csv"
in the same directory.

    \begin{Verbatim}[commandchars=\\\{\}]
{\color{incolor}In [{\color{incolor}68}]:} Data \PY{o}{\PYZlt{}\PYZhy{}} read.csv\PY{p}{(}file\PY{o}{=}\PY{l+s}{\PYZdq{}}\PY{l+s}{./FIT5197\PYZus{}2018\PYZus{}S1\PYZus{}Assignment1\PYZus{}Q2\PYZus{}data.csv\PYZdq{}}\PY{p}{,} header\PY{o}{=}\PY{k+kc}{TRUE}\PY{p}{,} sep\PY{o}{=}\PY{l+s}{\PYZdq{}}\PY{l+s}{,\PYZdq{}}\PY{p}{)}
\end{Verbatim}


    \begin{itemize}
\tightlist
\item
  Task 1. Handle \texttt{NA}s by mode imputation, and plot indivdual
  variables in a histogram with proper axis labels and title.
\end{itemize}

    \begin{itemize}
\tightlist
\item
  Task 2. Calculate and report full tables for \textbf{P(X)},
  \textbf{P(Y)}, \textbf{P(X, Y)}, \textbf{P(X\textbar{}Y)},
  \textbf{P(Y\textbar{}X)}.
\end{itemize}

    \begin{itemize}
\tightlist
\item
  Task 3. calculate and report \textbf{H(X)}, \textbf{H(Y)},
  \textbf{H(X\textbar{}Y)} and \textbf{H(Y\textbar{}X)}.
\end{itemize}


    % Add a bibliography block to the postdoc
    
    
    
    \end{document}
