\documentclass[]{article}
\usepackage{lmodern}
\usepackage{amssymb,amsmath}
\usepackage{ifxetex,ifluatex}
\usepackage{fixltx2e} % provides \textsubscript
\ifnum 0\ifxetex 1\fi\ifluatex 1\fi=0 % if pdftex
  \usepackage[T1]{fontenc}
  \usepackage[utf8]{inputenc}
\else % if luatex or xelatex
  \ifxetex
    \usepackage{mathspec}
  \else
    \usepackage{fontspec}
  \fi
  \defaultfontfeatures{Ligatures=TeX,Scale=MatchLowercase}
\fi
% use upquote if available, for straight quotes in verbatim environments
\IfFileExists{upquote.sty}{\usepackage{upquote}}{}
% use microtype if available
\IfFileExists{microtype.sty}{%
\usepackage{microtype}
\UseMicrotypeSet[protrusion]{basicmath} % disable protrusion for tt fonts
}{}
\usepackage[margin=1in]{geometry}
\usepackage{hyperref}
\hypersetup{unicode=true,
            pdftitle={FIT5197 2018 S1 Assignment 1},
            pdfauthor={Chuangfu Xie, 27771539},
            pdfborder={0 0 0},
            breaklinks=true}
\urlstyle{same}  % don't use monospace font for urls
\usepackage{color}
\usepackage{fancyvrb}
\newcommand{\VerbBar}{|}
\newcommand{\VERB}{\Verb[commandchars=\\\{\}]}
\DefineVerbatimEnvironment{Highlighting}{Verbatim}{commandchars=\\\{\}}
% Add ',fontsize=\small' for more characters per line
\usepackage{framed}
\definecolor{shadecolor}{RGB}{248,248,248}
\newenvironment{Shaded}{\begin{snugshade}}{\end{snugshade}}
\newcommand{\KeywordTok}[1]{\textcolor[rgb]{0.13,0.29,0.53}{\textbf{#1}}}
\newcommand{\DataTypeTok}[1]{\textcolor[rgb]{0.13,0.29,0.53}{#1}}
\newcommand{\DecValTok}[1]{\textcolor[rgb]{0.00,0.00,0.81}{#1}}
\newcommand{\BaseNTok}[1]{\textcolor[rgb]{0.00,0.00,0.81}{#1}}
\newcommand{\FloatTok}[1]{\textcolor[rgb]{0.00,0.00,0.81}{#1}}
\newcommand{\ConstantTok}[1]{\textcolor[rgb]{0.00,0.00,0.00}{#1}}
\newcommand{\CharTok}[1]{\textcolor[rgb]{0.31,0.60,0.02}{#1}}
\newcommand{\SpecialCharTok}[1]{\textcolor[rgb]{0.00,0.00,0.00}{#1}}
\newcommand{\StringTok}[1]{\textcolor[rgb]{0.31,0.60,0.02}{#1}}
\newcommand{\VerbatimStringTok}[1]{\textcolor[rgb]{0.31,0.60,0.02}{#1}}
\newcommand{\SpecialStringTok}[1]{\textcolor[rgb]{0.31,0.60,0.02}{#1}}
\newcommand{\ImportTok}[1]{#1}
\newcommand{\CommentTok}[1]{\textcolor[rgb]{0.56,0.35,0.01}{\textit{#1}}}
\newcommand{\DocumentationTok}[1]{\textcolor[rgb]{0.56,0.35,0.01}{\textbf{\textit{#1}}}}
\newcommand{\AnnotationTok}[1]{\textcolor[rgb]{0.56,0.35,0.01}{\textbf{\textit{#1}}}}
\newcommand{\CommentVarTok}[1]{\textcolor[rgb]{0.56,0.35,0.01}{\textbf{\textit{#1}}}}
\newcommand{\OtherTok}[1]{\textcolor[rgb]{0.56,0.35,0.01}{#1}}
\newcommand{\FunctionTok}[1]{\textcolor[rgb]{0.00,0.00,0.00}{#1}}
\newcommand{\VariableTok}[1]{\textcolor[rgb]{0.00,0.00,0.00}{#1}}
\newcommand{\ControlFlowTok}[1]{\textcolor[rgb]{0.13,0.29,0.53}{\textbf{#1}}}
\newcommand{\OperatorTok}[1]{\textcolor[rgb]{0.81,0.36,0.00}{\textbf{#1}}}
\newcommand{\BuiltInTok}[1]{#1}
\newcommand{\ExtensionTok}[1]{#1}
\newcommand{\PreprocessorTok}[1]{\textcolor[rgb]{0.56,0.35,0.01}{\textit{#1}}}
\newcommand{\AttributeTok}[1]{\textcolor[rgb]{0.77,0.63,0.00}{#1}}
\newcommand{\RegionMarkerTok}[1]{#1}
\newcommand{\InformationTok}[1]{\textcolor[rgb]{0.56,0.35,0.01}{\textbf{\textit{#1}}}}
\newcommand{\WarningTok}[1]{\textcolor[rgb]{0.56,0.35,0.01}{\textbf{\textit{#1}}}}
\newcommand{\AlertTok}[1]{\textcolor[rgb]{0.94,0.16,0.16}{#1}}
\newcommand{\ErrorTok}[1]{\textcolor[rgb]{0.64,0.00,0.00}{\textbf{#1}}}
\newcommand{\NormalTok}[1]{#1}
\usepackage{longtable,booktabs}
\usepackage{graphicx,grffile}
\makeatletter
\def\maxwidth{\ifdim\Gin@nat@width>\linewidth\linewidth\else\Gin@nat@width\fi}
\def\maxheight{\ifdim\Gin@nat@height>\textheight\textheight\else\Gin@nat@height\fi}
\makeatother
% Scale images if necessary, so that they will not overflow the page
% margins by default, and it is still possible to overwrite the defaults
% using explicit options in \includegraphics[width, height, ...]{}
\setkeys{Gin}{width=\maxwidth,height=\maxheight,keepaspectratio}
\IfFileExists{parskip.sty}{%
\usepackage{parskip}
}{% else
\setlength{\parindent}{0pt}
\setlength{\parskip}{6pt plus 2pt minus 1pt}
}
\setlength{\emergencystretch}{3em}  % prevent overfull lines
\providecommand{\tightlist}{%
  \setlength{\itemsep}{0pt}\setlength{\parskip}{0pt}}
\setcounter{secnumdepth}{0}
% Redefines (sub)paragraphs to behave more like sections
\ifx\paragraph\undefined\else
\let\oldparagraph\paragraph
\renewcommand{\paragraph}[1]{\oldparagraph{#1}\mbox{}}
\fi
\ifx\subparagraph\undefined\else
\let\oldsubparagraph\subparagraph
\renewcommand{\subparagraph}[1]{\oldsubparagraph{#1}\mbox{}}
\fi

%%% Use protect on footnotes to avoid problems with footnotes in titles
\let\rmarkdownfootnote\footnote%
\def\footnote{\protect\rmarkdownfootnote}

%%% Change title format to be more compact
\usepackage{titling}

% Create subtitle command for use in maketitle
\newcommand{\subtitle}[1]{
  \posttitle{
    \begin{center}\large#1\end{center}
    }
}

\setlength{\droptitle}{-2em}
  \title{FIT5197 2018 S1 Assignment 1}
  \pretitle{\vspace{\droptitle}\centering\huge}
  \posttitle{\par}
  \author{Chuangfu Xie, 27771539}
  \preauthor{\centering\large\emph}
  \postauthor{\par}
  \predate{\centering\large\emph}
  \postdate{\par}
  \date{13/04/2018}


\begin{document}
\maketitle

\subsection{Question 1: Calculate conditional probability of an
event}\label{question-1-calculate-conditional-probability-of-an-event}

\textbf{Solution}:\\
Let A denote the event that each value appears at least once, and B the
event that the outcome is alternate in numbers. To compute
P(A\textbar{}B), we have\\
\[ P(A|B)=\frac { P(AB) }{ P(B) } \] Firstly, we compute P(AB), to have
the intersection of event A and B:\\
1. In general, for a 7 time tossing, each tossing will yield a value
contributing to the outcome. Since a dice only have 6 possible values,
we have \[ { 6 }^{ 7 }=279936 \] 1. As event A state that each value
appears at least once, that is, to have a combination of 6 value without
duplication, we have \[ { A }_{ 6 }^{ 6 }=6!=720 \] 2. Then, we consider
event B which state that there are no two values are adjacent. To meet
this condition, as aforementioned permutation have 6 unique values, what
we do is selecting a value from 1 to 6 and inserting it into 7
``interval'' between these 6 values. we have
\[ (5+4+4+4+4+4+5)\div2=15 \]

For example, to insert a value right before the 1st value, the possible
value have to be in the 5 values of which have excluded the value of
following one, that is, 5 option for us to choose (Similar to inserting
a value behind the 6th value). For the ``interval'' where both side have
different values, on the analogy of previous reasoning, only 4 values
are available for us to pick. As in the combination, there are 7
``intervals'', only 5 ``interval'' are for this case.

\begin{longtable}[]{@{}lccccccr@{}}
\toprule
``Intervals'' & \({1}_{st}\) & \({2}_{nd}\) & \({3}_{rd}\) &
\({4}_{th}\) & \({5}_{th}\) & \({6}_{th}\) & \({7}_{th}\)\tabularnewline
\midrule
\endhead
options & 5 & 4 & 4 & 4 & 4 & 4 & 5\tabularnewline
\bottomrule
\end{longtable}

However, let's take a general situation as example: For 6 value
combination, says (1,2,3,4,5,6), inserting a value, which is identical
to the last value, into any ``interval'' from 1st to 5th, for instance,
(6,1,2,3,4,5,6), is exactly the same as the situation where inserting a
value right behind the tail of a given combination say (6,1,2,3,4,5),
which becomes (6,1,2,3,4,5,6) in the end.\\
Hence, the options we calculated so far have to be halved so that the
total possible options for us is 15.

To sum up, we have
\[ P(AB)=\frac { 15\times { A }_{ 6 }^{ 6 } }{ { 6 }^{ 7 } } =\frac { 15\times 720 }{ 279936 } =\frac { 25 }{ 648 }\approx 0.03858 \]

Next, we are going to compute P(B):\\
According to previous reasoning, if event B occurs, that is, no two
value adjacent having the same value. Hence, for each value in each
position, we denote each value as \$ \{ B \}\_\{ i \} \$ where i are the
position of the value. As values in each dicing are independent to
others, we have
\[ B={ B }_{ 1 }{ B }_{ 2 }{ B }_{ 3 }{ B }_{ 4 }{ B }_{ 5 }{ B }_{ 6 }{ B }_{ 7 } \]
then, we have
\[ P(B)=P({ B }_{ 1 }{ B }_{ 2 }{ B }_{ 3 }{ B }_{ 4 }{ B }_{ 5 }{ B }_{ 6 }{ B }_{ 7 })\]
\[ \Rightarrow P(B)=P({ B }_{ 1 }{ )\cdot P(B }_{ 2 }|{ B }_{ 1 }){ \cdot P(B }_{ 3 }|{ B }_{ 1 }{ B }_{ 2 }){ \cdot P(B }_{ 4 }|{ B }_{ 1 }{ B }_{ 2 }{ B }_{ 3 })\\ \quad \quad \quad \quad \quad { \cdot P(B }_{ 5 }|{ B }_{ 1 }{ B }_{ 2 }{ B }_{ 3 }{ B }_{ 4 }){ \cdot P(B }_{ 6 }|{ B }_{ 1 }{ B }_{ 2 }{ B }_{ 3 }{ B }_{ 4 }{ B }_{ 5 })\\ \quad \quad \quad \quad \quad { \cdot P(B }_{ 7 }|{ B }_{ 1 }{ B }_{ 2 }{ B }_{ 3 }{ B }_{ 4 }{ B }_{ 5 }{ B }_{ 6 }) \]
Since we have
\[ P({ B }_{ 1 })=\begin{pmatrix} 6 \\ 1 \end{pmatrix}\times \cfrac { 1 }{ 6 }=1 \]
\[ P({ B }_{ 2 }|{ B }_{ 1 })=\frac { P({ B }_{ 2 }{ B }_{ 1 }) }{ P({ B }_{ 1 })}=\cfrac { 1\times \cfrac { 5 }{ 6 }  }{ 1 } =\cfrac { 5 }{ 6 } \]
\[ P({ { B }_{ 3 }|B }_{ 2 }{ B }_{ 1 })=\frac { P({ { B }_{ 3 }B }_{ 2 }{ B }_{ 1 }) }{ P({ { B }_{ 2 }B }_{ 1 }) } =\cfrac { \cfrac { 5 }{ 6 } \times \cfrac { 5 }{ 6 }  }{ \cfrac { 5 }{ 6 }  } =\cfrac { 5 }{ 6 } \]
\[ \cdots \]
\[ { P(B }_{ 7 }|{ B }_{ 1 }{ B }_{ 2 }{ B }_{ 3 }{ B }_{ 4 }{ B }_{ 5 }{ B }_{ 6 })=\frac { P({ B }_{ 7 }{ { B }_{ 6 }{ B }_{ 5 } }{ B }_{ 4 }{ B }_{ 3 }{ B }_{ 2 }{ B }_{ 1 }) }{ P({ { B }_{ 6 }{ B }_{ 5 } }{ B }_{ 4 }{ B }_{ 3 }{ B }_{ 2 }{ B }_{ 1 }) } =\cfrac { \cfrac { 5 }{ 6 } \times \cfrac { 5 }{ 6 }  }{ \cfrac { 5 }{ 6 }  } =\cfrac { 5 }{ 6 } \]
To sum up
\[ { P(B })=1\times \cfrac { 5 }{ 6 } \times \cfrac { 5 }{ 6 } \times \cfrac { 5 }{ 6 } \times \cfrac { 5 }{ 6 } \times \cfrac { 5 }{ 6 } \times \cfrac { 5 }{ 6 } \approx 0.3349 \]

Finally, we can compute the P(A\textbar{}B):
\[ P(A|B)=\frac { P(AB) }{ P(B) }=\cfrac { \frac { 25 }{ 648 }  }{ 1\times \cfrac { 5 }{ 6 } \times \cfrac { 5 }{ 6 } \times \cfrac { 5 }{ 6 } \times \cfrac { 5 }{ 6 } \times \cfrac { 5 }{ 6 } \times \cfrac { 5 }{ 6 }  } = 0.1152 \]

\begin{Shaded}
\begin{Highlighting}[]
\NormalTok{nRuns =}\StringTok{ }\DecValTok{1000000} \CommentTok{#runs 1 million times}
\NormalTok{AB =}\StringTok{ }\DecValTok{0} \CommentTok{#initialise for counting event AB}
\NormalTok{B =}\StringTok{ }\DecValTok{0} \CommentTok{#initialise for counting event B}
\ControlFlowTok{for}\NormalTok{ (i }\ControlFlowTok{in} \DecValTok{1}\OperatorTok{:}\NormalTok{nRuns)\{}
    \CommentTok{# Tossing dice is discrete uniform distribution}
\NormalTok{    dice.roll <-}\StringTok{ }\KeywordTok{sample}\NormalTok{(}\KeywordTok{c}\NormalTok{(}\DecValTok{1}\OperatorTok{:}\DecValTok{6}\NormalTok{), }\DecValTok{7}\NormalTok{, }\DataTypeTok{replace =}\NormalTok{ T )}
    \CommentTok{# Check whether event A occurs}
\NormalTok{    A.result <-}\StringTok{ }\OperatorTok{!}\NormalTok{dice.roll }\OperatorTok\StringTok{ }\NormalTok{dice.roll[}\KeywordTok{duplicated}\NormalTok{(dice.roll)]}
    \CommentTok{# For computing p(AB)}
    \ControlFlowTok{if}\NormalTok{(}\KeywordTok{sum}\NormalTok{(A.result) }\OperatorTok{==}\StringTok{ }\DecValTok{5}\NormalTok{)\{}
\NormalTok{        AB.pre =}\StringTok{ }\DecValTok{0}
\NormalTok{        AB.cnt =}\StringTok{ }\DecValTok{0}
        \ControlFlowTok{for}\NormalTok{(i }\ControlFlowTok{in}\NormalTok{ dice.roll)\{}
            \ControlFlowTok{if}\NormalTok{ (AB.pre }\OperatorTok{==}\DecValTok{0}\NormalTok{)\{}
                \CommentTok{# For first loop}
                \CommentTok{# Initialize variable for previous value}
\NormalTok{                AB.pre <-}\StringTok{ }\NormalTok{i }
\NormalTok{            \}}\ControlFlowTok{else} \ControlFlowTok{if}\NormalTok{(}\KeywordTok{abs}\NormalTok{(pre}\OperatorTok{-}\NormalTok{i) }\OperatorTok{!=}\StringTok{ }\DecValTok{0}\NormalTok{)\{}
                \CommentTok{# The second if is for checking whether event B occurs by given event A occurs}
\NormalTok{                AB.cnt =}\StringTok{ }\NormalTok{AB.cnt }\OperatorTok{+}\StringTok{ }\DecValTok{1}
\NormalTok{            \}}
            \CommentTok{#Store previous value}
\NormalTok{            pre <-}\StringTok{ }\NormalTok{i}
\NormalTok{        \}}
        \ControlFlowTok{if}\NormalTok{(AB.cnt }\OperatorTok{==}\StringTok{ }\DecValTok{6}\NormalTok{)\{}
\NormalTok{            AB =}\StringTok{ }\NormalTok{AB }\OperatorTok{+}\StringTok{ }\DecValTok{1}
\NormalTok{        \}}
\NormalTok{    \}}
    \CommentTok{# For computing P(B)}
\NormalTok{    B.pre =}\StringTok{ }\DecValTok{0} 
\NormalTok{    B.cnt =}\StringTok{ }\DecValTok{0}
    \ControlFlowTok{for}\NormalTok{(i }\ControlFlowTok{in}\NormalTok{ dice.roll)\{}
        \CommentTok{# Check whether event B occurs}
        \ControlFlowTok{if}\NormalTok{ (B.pre }\OperatorTok{==}\DecValTok{0}\NormalTok{)\{}
\NormalTok{            B.pre <-}\StringTok{ }\NormalTok{i}
\NormalTok{        \}}\ControlFlowTok{else} \ControlFlowTok{if}\NormalTok{(}\KeywordTok{abs}\NormalTok{(pre}\OperatorTok{-}\NormalTok{i) }\OperatorTok{!=}\StringTok{ }\DecValTok{0}\NormalTok{)\{}
\NormalTok{            B.cnt =}\StringTok{ }\NormalTok{B.cnt }\OperatorTok{+}\StringTok{ }\DecValTok{1}
\NormalTok{        \}}
\NormalTok{        pre <-}\StringTok{ }\NormalTok{i }
\NormalTok{    \}}
    \ControlFlowTok{if}\NormalTok{(B.cnt }\OperatorTok{==}\StringTok{ }\DecValTok{6}\NormalTok{)\{}
\NormalTok{        B =}\StringTok{ }\NormalTok{B }\OperatorTok{+}\StringTok{ }\DecValTok{1}
\NormalTok{    \}}
\NormalTok{\}}
\KeywordTok{print}\NormalTok{(}\KeywordTok{paste}\NormalTok{(}\StringTok{"Simulation result:"}\NormalTok{,}\StringTok{"P(AB)="}\NormalTok{,AB}\OperatorTok{/}\NormalTok{nRuns,}\StringTok{", P(B)="}\NormalTok{,B}\OperatorTok{/}\NormalTok{nRuns,}\StringTok{", P(A|B)="}\NormalTok{,AB}\OperatorTok{/}\NormalTok{B))}
\end{Highlighting}
\end{Shaded}

\begin{verbatim}
## [1] "Simulation result: P(AB)= 0.038584 , P(B)= 0.335197 , P(A|B)= 0.115108428774721"
\end{verbatim}

\begin{center}\rule{0.5\linewidth}{\linethickness}\end{center}

\subsection{Question 2: Entropy}\label{question-2-entropy}

\textbf{Solution}:\\
First, read data from folder:\\
\textbf{Note}: To successfully read data from disk must have this
notebook and the csv file
``FIT5197\_2018\_S1\_Assignment1\_Q2\_data.csv'' in the same directory.

\begin{Shaded}
\begin{Highlighting}[]
\NormalTok{data <-}\StringTok{ }\KeywordTok{read.csv}\NormalTok{(}\DataTypeTok{file=}\StringTok{"./FIT5197_2018_S1_Assignment1_Q2_data.csv"}\NormalTok{, }\DataTypeTok{header=}\OtherTok{TRUE}\NormalTok{, }\DataTypeTok{sep=}\StringTok{","}\NormalTok{)}
\CommentTok{# Check how many NA in column X and Y}
\KeywordTok{print}\NormalTok{(}\KeywordTok{paste}\NormalTok{(}\StringTok{"X contains NAs in total: "}\NormalTok{,}\KeywordTok{sum}\NormalTok{(}\KeywordTok{is.na}\NormalTok{(data}\OperatorTok{$}\NormalTok{X))))}
\end{Highlighting}
\end{Shaded}

\begin{verbatim}
## [1] "X contains NAs in total:  10"
\end{verbatim}

\begin{Shaded}
\begin{Highlighting}[]
\KeywordTok{print}\NormalTok{(}\KeywordTok{paste}\NormalTok{(}\StringTok{"Y contains NAs in total: "}\NormalTok{,}\KeywordTok{sum}\NormalTok{(}\KeywordTok{is.na}\NormalTok{(data}\OperatorTok{$}\NormalTok{Y))))}
\end{Highlighting}
\end{Shaded}

\begin{verbatim}
## [1] "Y contains NAs in total:  13"
\end{verbatim}

\textbf{Task 1}. Handle \texttt{NA}s by mode imputation, and plot
indivdual variables in a histogram with proper axis labels and title.\\
\textbf{Solution}:\\
We use mode imputation to handle \texttt{NA} in our data, which is to
replace the missing value by its most common value.\\
First, we compute the mode:

\begin{Shaded}
\begin{Highlighting}[]
\KeywordTok{print}\NormalTok{(}\KeywordTok{paste}\NormalTok{(}\StringTok{"The mode in X is: "}\NormalTok{, }\KeywordTok{names}\NormalTok{(}\KeywordTok{which.max}\NormalTok{(}\KeywordTok{table}\NormalTok{(data}\OperatorTok{$}\NormalTok{X))), }\KeywordTok{paste}\NormalTok{(}\StringTok{"The mode in Y is: "}\NormalTok{, }\KeywordTok{names}\NormalTok{(}\KeywordTok{which.max}\NormalTok{(}\KeywordTok{table}\NormalTok{(data}\OperatorTok{$}\NormalTok{Y))))))}
\end{Highlighting}
\end{Shaded}

\begin{verbatim}
## [1] "The mode in X is:  0 The mode in Y is:  1"
\end{verbatim}

As the mode in X and Y, respectively, are 0 and 1, we assign the mode to
where there are \texttt{NA}s in X and Y :

\begin{Shaded}
\begin{Highlighting}[]
\NormalTok{data}\OperatorTok{$}\NormalTok{X[}\KeywordTok{is.na}\NormalTok{(data}\OperatorTok{$}\NormalTok{X)] <-}\StringTok{ }\DecValTok{0}
\NormalTok{data}\OperatorTok{$}\NormalTok{Y[}\KeywordTok{is.na}\NormalTok{(data}\OperatorTok{$}\NormalTok{Y)] <-}\StringTok{ }\DecValTok{1}
\end{Highlighting}
\end{Shaded}

Now we print the histogram:

\begin{Shaded}
\begin{Highlighting}[]
\KeywordTok{hist}\NormalTok{(data}\OperatorTok{$}\NormalTok{X, }\DataTypeTok{xlab=}\StringTok{"X"}\NormalTok{, }\DataTypeTok{breaks=}\DecValTok{3}\NormalTok{,}\DataTypeTok{col=}\KeywordTok{c}\NormalTok{(}\StringTok{"grey"}\NormalTok{,}\StringTok{"black"}\NormalTok{),}
     \DataTypeTok{labels=}\KeywordTok{c}\NormalTok{(}\KeywordTok{as.character}\NormalTok{(}\KeywordTok{sum}\NormalTok{(data}\OperatorTok{$}\NormalTok{X }\OperatorTok{==}\StringTok{ }\DecValTok{0}\NormalTok{)),}
              \KeywordTok{as.character}\NormalTok{(}\KeywordTok{sum}\NormalTok{(data}\OperatorTok{$}\NormalTok{X }\OperatorTok{==}\StringTok{ }\DecValTok{1}\NormalTok{))), }
     \DataTypeTok{xaxt=}\StringTok{'n'}\NormalTok{,}\DataTypeTok{main=}\StringTok{"Histogram of Event X"}\NormalTok{)}
\KeywordTok{axis}\NormalTok{(}\DataTypeTok{side =} \DecValTok{1}\NormalTok{,}\DataTypeTok{at=}\KeywordTok{seq}\NormalTok{(}\DecValTok{0}\NormalTok{,}\DecValTok{1}\NormalTok{,}\FloatTok{0.25}\NormalTok{),}\DataTypeTok{labels=}\KeywordTok{c}\NormalTok{(}\StringTok{""}\NormalTok{,}\DecValTok{0}\NormalTok{,}\StringTok{""}\NormalTok{,}\DecValTok{1}\NormalTok{,}\StringTok{""}\NormalTok{))}
\end{Highlighting}
\end{Shaded}

\includegraphics{ass1_files/figure-latex/unnamed-chunk-5-1.pdf}

\begin{Shaded}
\begin{Highlighting}[]
\KeywordTok{hist}\NormalTok{(data}\OperatorTok{$}\NormalTok{Y, }\DataTypeTok{xlab=}\StringTok{"Y"}\NormalTok{, }\DataTypeTok{breaks=}\DecValTok{3}\NormalTok{,}\DataTypeTok{col=}\KeywordTok{c}\NormalTok{(}\StringTok{"grey"}\NormalTok{,}\StringTok{"black"}\NormalTok{),}
     \DataTypeTok{labels=}\KeywordTok{c}\NormalTok{(}\KeywordTok{as.character}\NormalTok{(}\KeywordTok{sum}\NormalTok{(data}\OperatorTok{$}\NormalTok{Y }\OperatorTok{==}\StringTok{ }\DecValTok{0}\NormalTok{)),}
              \KeywordTok{as.character}\NormalTok{(}\KeywordTok{sum}\NormalTok{(data}\OperatorTok{$}\NormalTok{Y }\OperatorTok{==}\StringTok{ }\DecValTok{1}\NormalTok{))), }
     \DataTypeTok{xaxt=}\StringTok{'n'}\NormalTok{,}\DataTypeTok{main=}\StringTok{"Histogram of Event Y"}\NormalTok{)}
\KeywordTok{axis}\NormalTok{(}\DataTypeTok{side =} \DecValTok{1}\NormalTok{,}\DataTypeTok{at=}\KeywordTok{seq}\NormalTok{(}\DecValTok{0}\NormalTok{,}\DecValTok{1}\NormalTok{,}\FloatTok{0.25}\NormalTok{),}\DataTypeTok{labels=}\KeywordTok{c}\NormalTok{(}\StringTok{""}\NormalTok{,}\DecValTok{0}\NormalTok{,}\StringTok{""}\NormalTok{,}\DecValTok{1}\NormalTok{,}\StringTok{""}\NormalTok{))}
\end{Highlighting}
\end{Shaded}

\includegraphics{ass1_files/figure-latex/unnamed-chunk-5-2.pdf}

\begin{itemize}
\tightlist
\item
  \textbf{Task 2}. Calculate and report full tables for \textbf{P(X)},
  \textbf{P(Y)}, \textbf{P(X, Y)}, \textbf{P(X\textbar{}Y)},
  \textbf{P(Y\textbar{}X)}.\\
  \textbf{Solution}:\\
  Based on previous calculation, we already have all possible values of
  X and Y, which is the value table for each probability mass function,
  as we have \[ p(x_{i})=P_{ X }(X=x_{i}) \]
  \[ P_{ X }( 0 )=\frac { 59 }{ 100 } =0.59 \]
  \[ P_{ X }( 1 )=\frac { 41 }{ 100 } =0.41 \]
  \[ P_{ X }( 0 )+P_{ X }(1)=1\]
  \[ P_{ Y }( 0 )=\frac { 40 }{ 100 } =0.40 \]
  \[ P_{ Y }( 1 )=\frac { 60 }{ 100 } =0.60 \]
  \[ P_{ Y }( 0 )+P_{ Y }(1)=1\] X and Y both are discrete random
  variables, they also are mutual independent, thus we have:
  \[P(X,Y)=P(X)P(Y)\] \[ P( X|Y )=\frac { P(X,Y) }{ P(Y) } \]
  \[ P( Y|X )=\frac { P(X,Y) }{ P(X) } \] The result shown in table as
  follow:
\end{itemize}

Table \texttt{P(X)}:

\begin{longtable}[]{@{}lcr@{}}
\toprule
X & 0 & 1\tabularnewline
\midrule
\endhead
\texttt{p(x)} & 0.59 & 0.41\tabularnewline
\bottomrule
\end{longtable}

Table P(Y):

\begin{longtable}[]{@{}lcr@{}}
\toprule
Y & 0 & 1\tabularnewline
\midrule
\endhead
\texttt{p(x)} & 0.40 & 0.60\tabularnewline
\bottomrule
\end{longtable}

Table \texttt{P(X,Y)}:

\begin{longtable}[]{@{}lccr@{}}
\toprule
\texttt{Y\textbar{}X} & 0 & 1 & \texttt{P(x=j)}\tabularnewline
\midrule
\endhead
0 & 0.236 & 0.164 & 0.40\tabularnewline
1 & 0.354 & 0.246 & 0.60\tabularnewline
\texttt{P(x=i)} & 0.59 & 0.41 & 1\tabularnewline
\bottomrule
\end{longtable}

Table \texttt{P(X\textbar{}Y=0)}:

\begin{longtable}[]{@{}lcr@{}}
\toprule
X & 0 & 1\tabularnewline
\midrule
\endhead
\texttt{P(X\textbar{}Y=0)} & 0.575 & 0.425\tabularnewline
\bottomrule
\end{longtable}

Table \texttt{P(X\textbar{}Y=1)}:

\begin{longtable}[]{@{}lcr@{}}
\toprule
X & 0 & 1\tabularnewline
\midrule
\endhead
\texttt{P(X\textbar{}Y=1)} & 0.600 & 0.400\tabularnewline
\bottomrule
\end{longtable}

Table \texttt{P(Y\textbar{}X=0)}:

\begin{longtable}[]{@{}lcr@{}}
\toprule
Y & 0 & 1\tabularnewline
\midrule
\endhead
\texttt{P(Y\textbar{}X=0)} & 0.390 & 0.288\tabularnewline
\bottomrule
\end{longtable}

Table \texttt{P(Y\textbar{}X=1)}:

\begin{longtable}[]{@{}lcr@{}}
\toprule
Y & 0 & 1\tabularnewline
\midrule
\endhead
\texttt{P(Y\textbar{}X=1)} & 0.878 & 0.585\tabularnewline
\bottomrule
\end{longtable}

\begin{Shaded}
\begin{Highlighting}[]
\NormalTok{X_}\DecValTok{0}\NormalTok{ =}\StringTok{ }\KeywordTok{sum}\NormalTok{(data}\OperatorTok{$}\NormalTok{X }\OperatorTok{==}\StringTok{ }\DecValTok{0}\NormalTok{)}
\NormalTok{X_}\DecValTok{1}\NormalTok{ =}\StringTok{ }\KeywordTok{sum}\NormalTok{(data}\OperatorTok{$}\NormalTok{X }\OperatorTok{==}\StringTok{ }\DecValTok{1}\NormalTok{)}
\NormalTok{Y_}\DecValTok{0}\NormalTok{ =}\StringTok{ }\KeywordTok{sum}\NormalTok{(data}\OperatorTok{$}\NormalTok{Y }\OperatorTok{==}\StringTok{ }\DecValTok{0}\NormalTok{)}
\NormalTok{Y_}\DecValTok{1}\NormalTok{ =}\StringTok{ }\KeywordTok{sum}\NormalTok{(data}\OperatorTok{$}\NormalTok{Y }\OperatorTok{==}\StringTok{ }\DecValTok{1}\NormalTok{)}
\NormalTok{XY_}\DecValTok{00}\NormalTok{ =}\StringTok{ }\DecValTok{0}
\NormalTok{XY_}\DecValTok{01}\NormalTok{ =}\StringTok{ }\DecValTok{0}
\NormalTok{XY_}\DecValTok{10}\NormalTok{ =}\StringTok{ }\DecValTok{0}
\NormalTok{XY_}\DecValTok{11}\NormalTok{ =}\StringTok{ }\DecValTok{0}

\ControlFlowTok{for}\NormalTok{ (row }\ControlFlowTok{in} \DecValTok{1}\OperatorTok{:}\KeywordTok{nrow}\NormalTok{(data))\{}
\NormalTok{    x <-}\StringTok{ }\NormalTok{data[row,][[}\DecValTok{1}\NormalTok{]]}
\NormalTok{    y <-}\StringTok{ }\NormalTok{data[row,][[}\DecValTok{2}\NormalTok{]]}
    \CommentTok{# counting F(X,Y)}
    \ControlFlowTok{if}\NormalTok{(x}\OperatorTok{==}\DecValTok{0}\OperatorTok{&}\NormalTok{y}\OperatorTok{==}\DecValTok{0}\NormalTok{)\{}
\NormalTok{        XY_}\DecValTok{00}\NormalTok{=XY_}\DecValTok{00}\OperatorTok{+}\DecValTok{1}
\NormalTok{    \}}\ControlFlowTok{else} \ControlFlowTok{if}\NormalTok{(x}\OperatorTok{==}\DecValTok{0}\OperatorTok{&}\NormalTok{y}\OperatorTok{==}\DecValTok{1}\NormalTok{)\{}
\NormalTok{        XY_}\DecValTok{01}\NormalTok{=XY_}\DecValTok{01}\OperatorTok{+}\DecValTok{1}
\NormalTok{    \}}\ControlFlowTok{else} \ControlFlowTok{if}\NormalTok{(x}\OperatorTok{==}\DecValTok{1}\OperatorTok{&}\NormalTok{y}\OperatorTok{==}\DecValTok{0}\NormalTok{)\{}
\NormalTok{        XY_}\DecValTok{10}\NormalTok{=XY_}\DecValTok{10}\OperatorTok{+}\DecValTok{1}
\NormalTok{    \}}\ControlFlowTok{else} \ControlFlowTok{if}\NormalTok{(x}\OperatorTok{==}\DecValTok{1}\OperatorTok{&}\NormalTok{y}\OperatorTok{==}\DecValTok{1}\NormalTok{)\{}
\NormalTok{        XY_}\DecValTok{11}\NormalTok{=XY_}\DecValTok{11}\OperatorTok{+}\DecValTok{1}
\NormalTok{    \}}
\NormalTok{\}}
\KeywordTok{print}\NormalTok{(}\KeywordTok{paste}\NormalTok{(}\StringTok{"P(X,Y)=P(0,0)="}\NormalTok{, XY_}\DecValTok{00}\OperatorTok{/}\DecValTok{100}\NormalTok{))}
\end{Highlighting}
\end{Shaded}

\begin{verbatim}
## [1] "P(X,Y)=P(0,0)= 0.23"
\end{verbatim}

\begin{Shaded}
\begin{Highlighting}[]
\KeywordTok{print}\NormalTok{(}\KeywordTok{paste}\NormalTok{(}\StringTok{"P(X,Y)=P(1,0)="}\NormalTok{, XY_}\DecValTok{10}\OperatorTok{/}\DecValTok{100}\NormalTok{))}
\end{Highlighting}
\end{Shaded}

\begin{verbatim}
## [1] "P(X,Y)=P(1,0)= 0.17"
\end{verbatim}

\begin{Shaded}
\begin{Highlighting}[]
\KeywordTok{print}\NormalTok{(}\KeywordTok{paste}\NormalTok{(}\StringTok{"P(X,Y)=P(0,1)="}\NormalTok{, XY_}\DecValTok{01}\OperatorTok{/}\DecValTok{100}\NormalTok{))}
\end{Highlighting}
\end{Shaded}

\begin{verbatim}
## [1] "P(X,Y)=P(0,1)= 0.36"
\end{verbatim}

\begin{Shaded}
\begin{Highlighting}[]
\KeywordTok{print}\NormalTok{(}\KeywordTok{paste}\NormalTok{(}\StringTok{"P(X,Y)=P(1,1)="}\NormalTok{, XY_}\DecValTok{11}\OperatorTok{/}\DecValTok{100}\NormalTok{))}
\end{Highlighting}
\end{Shaded}

\begin{verbatim}
## [1] "P(X,Y)=P(1,1)= 0.24"
\end{verbatim}

\begin{Shaded}
\begin{Highlighting}[]
\KeywordTok{print}\NormalTok{(}\KeywordTok{paste}\NormalTok{(}\StringTok{"P(X|Y)=P(0|0)="}\NormalTok{, XY_}\DecValTok{00}\OperatorTok{/}\NormalTok{Y_}\DecValTok{0}\NormalTok{))}
\end{Highlighting}
\end{Shaded}

\begin{verbatim}
## [1] "P(X|Y)=P(0|0)= 0.575"
\end{verbatim}

\begin{Shaded}
\begin{Highlighting}[]
\KeywordTok{print}\NormalTok{(}\KeywordTok{paste}\NormalTok{(}\StringTok{"P(X|Y)=P(1|0)="}\NormalTok{, XY_}\DecValTok{10}\OperatorTok{/}\NormalTok{Y_}\DecValTok{0}\NormalTok{))}
\end{Highlighting}
\end{Shaded}

\begin{verbatim}
## [1] "P(X|Y)=P(1|0)= 0.425"
\end{verbatim}

\begin{Shaded}
\begin{Highlighting}[]
\KeywordTok{print}\NormalTok{(}\KeywordTok{paste}\NormalTok{(}\StringTok{"P(X|Y)=P(0|1)="}\NormalTok{, XY_}\DecValTok{01}\OperatorTok{/}\NormalTok{Y_}\DecValTok{1}\NormalTok{))}
\end{Highlighting}
\end{Shaded}

\begin{verbatim}
## [1] "P(X|Y)=P(0|1)= 0.6"
\end{verbatim}

\begin{Shaded}
\begin{Highlighting}[]
\KeywordTok{print}\NormalTok{(}\KeywordTok{paste}\NormalTok{(}\StringTok{"P(X|Y)=P(1|1)="}\NormalTok{, XY_}\DecValTok{11}\OperatorTok{/}\NormalTok{Y_}\DecValTok{1}\NormalTok{))}
\end{Highlighting}
\end{Shaded}

\begin{verbatim}
## [1] "P(X|Y)=P(1|1)= 0.4"
\end{verbatim}

\begin{Shaded}
\begin{Highlighting}[]
\KeywordTok{print}\NormalTok{(}\KeywordTok{paste}\NormalTok{(}\StringTok{"P(Y|X)=P(0|0)="}\NormalTok{, XY_}\DecValTok{00}\OperatorTok{/}\NormalTok{X_}\DecValTok{0}\NormalTok{))}
\end{Highlighting}
\end{Shaded}

\begin{verbatim}
## [1] "P(Y|X)=P(0|0)= 0.389830508474576"
\end{verbatim}

\begin{Shaded}
\begin{Highlighting}[]
\KeywordTok{print}\NormalTok{(}\KeywordTok{paste}\NormalTok{(}\StringTok{"P(Y|X)=P(1|0)="}\NormalTok{, XY_}\DecValTok{10}\OperatorTok{/}\NormalTok{X_}\DecValTok{0}\NormalTok{))}
\end{Highlighting}
\end{Shaded}

\begin{verbatim}
## [1] "P(Y|X)=P(1|0)= 0.288135593220339"
\end{verbatim}

\begin{Shaded}
\begin{Highlighting}[]
\KeywordTok{print}\NormalTok{(}\KeywordTok{paste}\NormalTok{(}\StringTok{"P(Y|X)=P(0|1)="}\NormalTok{, XY_}\DecValTok{01}\OperatorTok{/}\NormalTok{X_}\DecValTok{1}\NormalTok{))}
\end{Highlighting}
\end{Shaded}

\begin{verbatim}
## [1] "P(Y|X)=P(0|1)= 0.878048780487805"
\end{verbatim}

\begin{Shaded}
\begin{Highlighting}[]
\KeywordTok{print}\NormalTok{(}\KeywordTok{paste}\NormalTok{(}\StringTok{"P(Y|X)=P(1|1)="}\NormalTok{, XY_}\DecValTok{11}\OperatorTok{/}\NormalTok{X_}\DecValTok{1}\NormalTok{))}
\end{Highlighting}
\end{Shaded}

\begin{verbatim}
## [1] "P(Y|X)=P(1|1)= 0.585365853658537"
\end{verbatim}

\begin{center}\rule{0.5\linewidth}{\linethickness}\end{center}

\begin{itemize}
\tightlist
\item
  \textbf{Task 3}. calculate and report \textbf{H(X)}, \textbf{H(Y)},
  \textbf{H(X\textbar{}Y)} and \textbf{H(Y\textbar{}X)}.\\
  \textbf{Solution}:\\
  Sice we have
  \[H(X)=-\sum _{ i=1 }^{ n }{ p_{ i }\log _{ 2 }{ ({ p }_{ i }) }  } \]
  All results shown as follow: \[ H(X)=H(X=0)+H(X=1)\approx0.9765\]
  \[ H(X)=H(Y=0)+H(Y=1)\approx0.9709\]
  \[ H(X|Y)=H(0|0)+H(1|0)+H(0|1)+H(1|1)\approx1.9546\]
  \[H(Y|X)=H(0|0)+H(1|0)+H(0|1)+H(1|1)\approx1.6641\]
\end{itemize}

\begin{Shaded}
\begin{Highlighting}[]
\KeywordTok{print}\NormalTok{(}\KeywordTok{paste}\NormalTok{(}\StringTok{"H(X)="}\NormalTok{, }\OperatorTok{-}\DecValTok{1}\OperatorTok{*}\NormalTok{((X_}\DecValTok{0}\OperatorTok{/}\DecValTok{100}\NormalTok{)}\OperatorTok{*}\KeywordTok{log2}\NormalTok{(X_}\DecValTok{0}\OperatorTok{/}\DecValTok{100}\NormalTok{)}\OperatorTok{+}\NormalTok{(X_}\DecValTok{1}\OperatorTok{/}\DecValTok{100}\NormalTok{)}\OperatorTok{*}\KeywordTok{log2}\NormalTok{(X_}\DecValTok{1}\OperatorTok{/}\DecValTok{100}\NormalTok{))))}
\end{Highlighting}
\end{Shaded}

\begin{verbatim}
## [1] "H(X)= 0.976500468757824"
\end{verbatim}

\begin{Shaded}
\begin{Highlighting}[]
\KeywordTok{print}\NormalTok{(}\KeywordTok{paste}\NormalTok{(}\StringTok{"H(Y)="}\NormalTok{, }\OperatorTok{-}\DecValTok{1}\OperatorTok{*}\NormalTok{((Y_}\DecValTok{0}\OperatorTok{/}\DecValTok{100}\NormalTok{)}\OperatorTok{*}\KeywordTok{log2}\NormalTok{(Y_}\DecValTok{0}\OperatorTok{/}\DecValTok{100}\NormalTok{)}\OperatorTok{+}\NormalTok{(Y_}\DecValTok{1}\OperatorTok{/}\DecValTok{100}\NormalTok{)}\OperatorTok{*}\KeywordTok{log2}\NormalTok{(Y_}\DecValTok{1}\OperatorTok{/}\DecValTok{100}\NormalTok{))))}
\end{Highlighting}
\end{Shaded}

\begin{verbatim}
## [1] "H(Y)= 0.970950594454669"
\end{verbatim}

\begin{Shaded}
\begin{Highlighting}[]
\KeywordTok{print}\NormalTok{(}\KeywordTok{paste}\NormalTok{(}\StringTok{"H(X|Y)="}\NormalTok{, }
            \OperatorTok{-}\DecValTok{1}\OperatorTok{*}\NormalTok{((XY_}\DecValTok{00}\OperatorTok{/}\NormalTok{Y_}\DecValTok{0}\NormalTok{)}\OperatorTok{*}\KeywordTok{log2}\NormalTok{(XY_}\DecValTok{00}\OperatorTok{/}\NormalTok{Y_}\DecValTok{0}\NormalTok{)}\OperatorTok{+}
\StringTok{            }\NormalTok{(XY_}\DecValTok{10}\OperatorTok{/}\NormalTok{Y_}\DecValTok{0}\NormalTok{)}\OperatorTok{*}\KeywordTok{log2}\NormalTok{(XY_}\DecValTok{10}\OperatorTok{/}\NormalTok{Y_}\DecValTok{0}\NormalTok{)}\OperatorTok{+}
\StringTok{            }\NormalTok{(XY_}\DecValTok{01}\OperatorTok{/}\NormalTok{Y_}\DecValTok{1}\NormalTok{)}\OperatorTok{*}\KeywordTok{log2}\NormalTok{(XY_}\DecValTok{01}\OperatorTok{/}\NormalTok{Y_}\DecValTok{1}\NormalTok{)}\OperatorTok{+}
\StringTok{            }\NormalTok{(XY_}\DecValTok{11}\OperatorTok{/}\NormalTok{Y_}\DecValTok{1}\NormalTok{)}\OperatorTok{*}\KeywordTok{log2}\NormalTok{(XY_}\DecValTok{11}\OperatorTok{/}\NormalTok{Y_}\DecValTok{1}\NormalTok{))))}
\end{Highlighting}
\end{Shaded}

\begin{verbatim}
## [1] "H(X|Y)= 1.95465885707785"
\end{verbatim}

\begin{Shaded}
\begin{Highlighting}[]
\KeywordTok{print}\NormalTok{(}\KeywordTok{paste}\NormalTok{(}\StringTok{"H(Y|X)="}\NormalTok{, }
            \OperatorTok{-}\DecValTok{1}\OperatorTok{*}\NormalTok{((XY_}\DecValTok{00}\OperatorTok{/}\NormalTok{X_}\DecValTok{0}\NormalTok{)}\OperatorTok{*}\KeywordTok{log2}\NormalTok{(XY_}\DecValTok{00}\OperatorTok{/}\NormalTok{X_}\DecValTok{0}\NormalTok{)}\OperatorTok{+}
\StringTok{            }\NormalTok{(XY_}\DecValTok{10}\OperatorTok{/}\NormalTok{X_}\DecValTok{0}\NormalTok{)}\OperatorTok{*}\KeywordTok{log2}\NormalTok{(XY_}\DecValTok{10}\OperatorTok{/}\NormalTok{X_}\DecValTok{0}\NormalTok{)}\OperatorTok{+}
\StringTok{            }\NormalTok{(XY_}\DecValTok{01}\OperatorTok{/}\NormalTok{X_}\DecValTok{1}\NormalTok{)}\OperatorTok{*}\KeywordTok{log2}\NormalTok{(XY_}\DecValTok{01}\OperatorTok{/}\NormalTok{X_}\DecValTok{1}\NormalTok{)}\OperatorTok{+}
\StringTok{            }\NormalTok{(XY_}\DecValTok{11}\OperatorTok{/}\NormalTok{X_}\DecValTok{1}\NormalTok{)}\OperatorTok{*}\KeywordTok{log2}\NormalTok{(XY_}\DecValTok{11}\OperatorTok{/}\NormalTok{X_}\DecValTok{1}\NormalTok{))))}
\end{Highlighting}
\end{Shaded}

\begin{verbatim}
## [1] "H(Y|X)= 1.66405976366414"
\end{verbatim}

\begin{center}\rule{0.5\linewidth}{\linethickness}\end{center}

\subsection{Question 3: Correlation and
covariance}\label{question-3-correlation-and-covariance}

\textbf{Solution}: As X and Y are two standard Gaussian random
variables, we have
\[ X\sim N(\mu_{X} ,{ \sigma_{X}  }^{ 2 })\Rightarrow X\sim N(0,1)\]
\[ Y\sim N(\mu_{Y} ,{ \sigma_{Y}  }^{ 2 })\Rightarrow Y\sim N(0,1)\]
Then \[ E(X)=0, D(X)=1 \] \[ E(Y)=0, D(Y)=1 \] Since X and Y are mutual
independent, as we have U = X - Y adn V = 2X +3Y, we can infer that U
and V are also Gaussian distributon, we have
\[ { C }_{ 1 }{ X }_{ 1 }+{ C }_{ 2 }{ X }_{ 2 }+\cdots +{ C }_{ n }{ X }_{ n }\sim N(\sum _{ i=1 }^{ n }{ { C }_{ i }\mu _{ i } } ,\sum _{ i=1 }^{ n }{ { C }_{ i }^{ 2 }{ \sigma  }_{ i }^{ 2 } } ) \]
Then \[ U=X-Y\sim N(0,1) \] \[ V=2X+3Y\sim N(0,13) \]
\[ E(U)=0, D(X)=1 \] \[ E(V)=0, D(Y)=13 \] As we need to justify what is
the correlation and covariance between U and V, first we calculate the
covariance, we have
\[ Cov(U,V)=E\{ [U-E(U)][V-E(V)]\} \\ =E\{ [X-Y-E(U)][2X+3Y-E(V)]\} \\ =E\{ 2{ X }^{ 2 }+3{ Y }^{ 2 }+XY-E(U)E(V)\} \\ =E(XY)-E(U)E(V)\\ =E(X)E(Y)-E(U)E(V)=0 \]
If \texttt{Cov(U,V)=0} that means U and V have no correlation

\begin{Shaded}
\begin{Highlighting}[]
\NormalTok{nRuns =}\StringTok{ }\DecValTok{100000}
\NormalTok{total_cor =}\StringTok{ }\DecValTok{0}
\NormalTok{total_cov =}\StringTok{ }\DecValTok{0}
\ControlFlowTok{for}\NormalTok{ (i }\ControlFlowTok{in} \DecValTok{1}\OperatorTok{:}\NormalTok{nRuns)\{}
\NormalTok{    U <-}\KeywordTok{rnorm}\NormalTok{(}\DecValTok{1000}\NormalTok{,}\DataTypeTok{mean=}\DecValTok{0}\NormalTok{,}\DataTypeTok{sd=}\DecValTok{1}\NormalTok{)}
\NormalTok{    V <-}\KeywordTok{rnorm}\NormalTok{(}\DecValTok{1000}\NormalTok{,}\DataTypeTok{mean=}\DecValTok{0}\NormalTok{,}\DataTypeTok{sd=}\DecValTok{13}\NormalTok{)}
\NormalTok{    total_cor =}\StringTok{ }\NormalTok{total_cor }\OperatorTok{+}\StringTok{ }\KeywordTok{cor}\NormalTok{(U,V)}
\NormalTok{    total_cov =}\StringTok{ }\NormalTok{total_cov }\OperatorTok{+}\StringTok{ }\KeywordTok{cov}\NormalTok{(U,V)}
\NormalTok{\}}
\NormalTok{cor_result <-}\StringTok{ }\NormalTok{total_cor}\OperatorTok{/}\NormalTok{nRuns}
\NormalTok{cov_result <-}\StringTok{ }\NormalTok{total_cov}\OperatorTok{/}\NormalTok{nRuns}
\KeywordTok{print}\NormalTok{(}\KeywordTok{paste}\NormalTok{(}\StringTok{"Correlation:"}\NormalTok{,cor_result))}
\end{Highlighting}
\end{Shaded}

\begin{verbatim}
## [1] "Correlation: 0.000137364060082424"
\end{verbatim}

\begin{Shaded}
\begin{Highlighting}[]
\KeywordTok{print}\NormalTok{(}\KeywordTok{paste}\NormalTok{(}\StringTok{"Covariance:"}\NormalTok{,cov_result))}
\end{Highlighting}
\end{Shaded}

\begin{verbatim}
## [1] "Covariance: 0.00181325605474258"
\end{verbatim}

\begin{center}\rule{0.5\linewidth}{\linethickness}\end{center}

\subsection{Question 4: Maximum likelihood estimation of
parameters}\label{question-4-maximum-likelihood-estimation-of-parameters}

\textbf{Solution}:\\
As data \texttt{{[}4,3,2,4,6,3,4,0,5,6,4,4,4,5,3,3,4,5,4,5{]}} comes
from a Poisson distribution with unknown parameter λ, we found the
parameter by its likelihood function:
\[\log { f({ x }_{ 1 },{ x }_{ 2 },\dots ,{ x }_{ n }|\lambda ) } =-n\lambda +\sum _{ 1 }^{ n }{ { x }_{ i }\log { \lambda  }  } -\sum _{ 1 }^{ n }{ \log { { x }_{ i }! }  } \]
Hence

\begin{Shaded}
\begin{Highlighting}[]
\NormalTok{P_data <-}\StringTok{ }\KeywordTok{c}\NormalTok{(}\DecValTok{4}\NormalTok{,}\DecValTok{3}\NormalTok{,}\DecValTok{2}\NormalTok{,}\DecValTok{4}\NormalTok{,}\DecValTok{6}\NormalTok{,}\DecValTok{3}\NormalTok{,}\DecValTok{4}\NormalTok{,}\DecValTok{0}\NormalTok{,}\DecValTok{5}\NormalTok{,}\DecValTok{6}\NormalTok{,}\DecValTok{4}\NormalTok{,}\DecValTok{4}\NormalTok{,}\DecValTok{4}\NormalTok{,}\DecValTok{5}\NormalTok{,}\DecValTok{3}\NormalTok{,}\DecValTok{3}\NormalTok{,}\DecValTok{4}\NormalTok{,}\DecValTok{5}\NormalTok{,}\DecValTok{4}\NormalTok{,}\DecValTok{5}\NormalTok{)}
\NormalTok{L <-}\StringTok{ }\ControlFlowTok{function}\NormalTok{(lambda, data)\{}
\NormalTok{    x_list <-}\StringTok{ }\KeywordTok{as.double}\NormalTok{(}\KeywordTok{names}\NormalTok{(}\KeywordTok{table}\NormalTok{(P_data)))}
\NormalTok{    n <-}\StringTok{ }\KeywordTok{length}\NormalTok{(x_list)}
\NormalTok{    sum_x =}\StringTok{ }\DecValTok{0}
\NormalTok{    sum_log_factorial =}\StringTok{ }\DecValTok{0}
\NormalTok{    result =}\StringTok{ }\DecValTok{0}
    \CommentTok{# compute all sums}
    \ControlFlowTok{for}\NormalTok{ (i }\ControlFlowTok{in} \DecValTok{1}\OperatorTok{:}\NormalTok{n)\{}
\NormalTok{        x =}\StringTok{ }\NormalTok{x_list[i]}
\NormalTok{        sum_x =}\StringTok{ }\NormalTok{sum_x }\OperatorTok{+}\StringTok{ }\NormalTok{x}
\NormalTok{        sum_log_factorial =}\StringTok{ }\NormalTok{sum_log_factorial }\OperatorTok{+}\StringTok{ }\KeywordTok{log10}\NormalTok{(}\KeywordTok{factorial}\NormalTok{(x))}
\NormalTok{    \}}
\NormalTok{    result =}\StringTok{ }\NormalTok{sum_x}\OperatorTok{*}\KeywordTok{log10}\NormalTok{(lambda) }\OperatorTok{-}\StringTok{ }\NormalTok{sum_log_factorial }\OperatorTok{-}\StringTok{ }\NormalTok{n}\OperatorTok{*}\NormalTok{lambda}
    \KeywordTok{print}\NormalTok{(result)}
    \KeywordTok{return}\NormalTok{(result)}
\NormalTok{\}}
\KeywordTok{optimize}\NormalTok{(L, }\DataTypeTok{interval=}\KeywordTok{c}\NormalTok{(}\DecValTok{0}\NormalTok{,}\DecValTok{10}\NormalTok{),}\DataTypeTok{tol=}\FloatTok{0.001}\NormalTok{, }\DataTypeTok{data=}\NormalTok{P_data, }\DataTypeTok{maximum =}\NormalTok{ T)}
\end{Highlighting}
\end{Shaded}

\begin{verbatim}
## [1] -18.67337
## [1] -28.6577
## [1] -14.09924
## [1] -12.8688
## [1] -13.39553
## [1] -12.88563
## [1] -12.86878
## [1] -12.86854
## [1] -12.93456
## [1] -12.86853
## [1] -12.86853
## [1] -12.86853
\end{verbatim}

\begin{verbatim}
## $maximum
## [1] 1.447623
## 
## $objective
## [1] -12.86853
\end{verbatim}

As the result shown \[ \hat { \lambda  } \approx 1.447 \] Also,we do the
point estimation to check:

\begin{Shaded}
\begin{Highlighting}[]
\CommentTok{#Point Estimation}
\NormalTok{PE <-}\StringTok{ }\ControlFlowTok{function}\NormalTok{(data)\{}
\NormalTok{    cnt <-}\StringTok{ }\KeywordTok{table}\NormalTok{(data)}
\NormalTok{    n <-}\StringTok{ }\KeywordTok{length}\NormalTok{(data)}
\NormalTok{    E =}\StringTok{ }\DecValTok{0}
    \ControlFlowTok{for}\NormalTok{ (i }\ControlFlowTok{in} \DecValTok{1}\OperatorTok{:}\KeywordTok{nrow}\NormalTok{(cnt))\{}
        \KeywordTok{print}\NormalTok{(}\KeywordTok{paste}\NormalTok{(}\StringTok{"value"}\NormalTok{,i,}\StringTok{"="}\NormalTok{,}\KeywordTok{as.double}\NormalTok{(}\KeywordTok{names}\NormalTok{(cnt)[i]),}\StringTok{","}\NormalTok{,}\StringTok{"P("}\NormalTok{,i,}\StringTok{")="}\NormalTok{,}\KeywordTok{as.double}\NormalTok{(cnt[[i]])}\OperatorTok{/}\NormalTok{n))}
        \KeywordTok{print}\NormalTok{(}\KeywordTok{paste}\NormalTok{(}\StringTok{"E(X="}\NormalTok{,}\KeywordTok{as.double}\NormalTok{(}\KeywordTok{names}\NormalTok{(cnt)[i]), }\StringTok{")= "}\NormalTok{,}\KeywordTok{as.double}\NormalTok{(}\KeywordTok{names}\NormalTok{(cnt)[i])}\OperatorTok{*}\KeywordTok{as.double}\NormalTok{(cnt[[i]])}\OperatorTok{/}\NormalTok{n))}
\NormalTok{        E =}\StringTok{ }\NormalTok{E }\OperatorTok{+}\StringTok{ }\KeywordTok{as.double}\NormalTok{(}\KeywordTok{names}\NormalTok{(cnt)[i])}\OperatorTok{*}\KeywordTok{as.double}\NormalTok{(cnt[[i]])}\OperatorTok{/}\NormalTok{n}
\NormalTok{    \}}
    \KeywordTok{print}\NormalTok{(}\KeywordTok{paste}\NormalTok{(}\StringTok{"E(X)="}\NormalTok{,E))}
\NormalTok{\}}
\KeywordTok{PE}\NormalTok{(P_data)}
\end{Highlighting}
\end{Shaded}

\begin{verbatim}
## [1] "value 1 = 0 , P( 1 )= 0.05"
## [1] "E(X= 0 )=  0"
## [1] "value 2 = 2 , P( 2 )= 0.05"
## [1] "E(X= 2 )=  0.1"
## [1] "value 3 = 3 , P( 3 )= 0.2"
## [1] "E(X= 3 )=  0.6"
## [1] "value 4 = 4 , P( 4 )= 0.4"
## [1] "E(X= 4 )=  1.6"
## [1] "value 5 = 5 , P( 5 )= 0.2"
## [1] "E(X= 5 )=  1"
## [1] "value 6 = 6 , P( 6 )= 0.1"
## [1] "E(X= 6 )=  0.6"
## [1] "E(X)= 3.9"
\end{verbatim}


\end{document}
