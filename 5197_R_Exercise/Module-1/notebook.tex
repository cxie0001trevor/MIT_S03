
% Default to the notebook output style

    


% Inherit from the specified cell style.




    
\documentclass[11pt]{article}

    
    
    \usepackage[T1]{fontenc}
    % Nicer default font (+ math font) than Computer Modern for most use cases
    \usepackage{mathpazo}

    % Basic figure setup, for now with no caption control since it's done
    % automatically by Pandoc (which extracts ![](path) syntax from Markdown).
    \usepackage{graphicx}
    % We will generate all images so they have a width \maxwidth. This means
    % that they will get their normal width if they fit onto the page, but
    % are scaled down if they would overflow the margins.
    \makeatletter
    \def\maxwidth{\ifdim\Gin@nat@width>\linewidth\linewidth
    \else\Gin@nat@width\fi}
    \makeatother
    \let\Oldincludegraphics\includegraphics
    % Set max figure width to be 80% of text width, for now hardcoded.
    \renewcommand{\includegraphics}[1]{\Oldincludegraphics[width=.8\maxwidth]{#1}}
    % Ensure that by default, figures have no caption (until we provide a
    % proper Figure object with a Caption API and a way to capture that
    % in the conversion process - todo).
    \usepackage{caption}
    \DeclareCaptionLabelFormat{nolabel}{}
    \captionsetup{labelformat=nolabel}

    \usepackage{adjustbox} % Used to constrain images to a maximum size 
    \usepackage{xcolor} % Allow colors to be defined
    \usepackage{enumerate} % Needed for markdown enumerations to work
    \usepackage{geometry} % Used to adjust the document margins
    \usepackage{amsmath} % Equations
    \usepackage{amssymb} % Equations
    \usepackage{textcomp} % defines textquotesingle
    % Hack from http://tex.stackexchange.com/a/47451/13684:
    \AtBeginDocument{%
        \def\PYZsq{\textquotesingle}% Upright quotes in Pygmentized code
    }
    \usepackage{upquote} % Upright quotes for verbatim code
    \usepackage{eurosym} % defines \euro
    \usepackage[mathletters]{ucs} % Extended unicode (utf-8) support
    \usepackage[utf8x]{inputenc} % Allow utf-8 characters in the tex document
    \usepackage{fancyvrb} % verbatim replacement that allows latex
    \usepackage{grffile} % extends the file name processing of package graphics 
                         % to support a larger range 
    % The hyperref package gives us a pdf with properly built
    % internal navigation ('pdf bookmarks' for the table of contents,
    % internal cross-reference links, web links for URLs, etc.)
    \usepackage{hyperref}
    \usepackage{longtable} % longtable support required by pandoc >1.10
    \usepackage{booktabs}  % table support for pandoc > 1.12.2
    \usepackage[inline]{enumitem} % IRkernel/repr support (it uses the enumerate* environment)
    \usepackage[normalem]{ulem} % ulem is needed to support strikethroughs (\sout)
                                % normalem makes italics be italics, not underlines
    

    
    
    % Colors for the hyperref package
    \definecolor{urlcolor}{rgb}{0,.145,.698}
    \definecolor{linkcolor}{rgb}{.71,0.21,0.01}
    \definecolor{citecolor}{rgb}{.12,.54,.11}

    % ANSI colors
    \definecolor{ansi-black}{HTML}{3E424D}
    \definecolor{ansi-black-intense}{HTML}{282C36}
    \definecolor{ansi-red}{HTML}{E75C58}
    \definecolor{ansi-red-intense}{HTML}{B22B31}
    \definecolor{ansi-green}{HTML}{00A250}
    \definecolor{ansi-green-intense}{HTML}{007427}
    \definecolor{ansi-yellow}{HTML}{DDB62B}
    \definecolor{ansi-yellow-intense}{HTML}{B27D12}
    \definecolor{ansi-blue}{HTML}{208FFB}
    \definecolor{ansi-blue-intense}{HTML}{0065CA}
    \definecolor{ansi-magenta}{HTML}{D160C4}
    \definecolor{ansi-magenta-intense}{HTML}{A03196}
    \definecolor{ansi-cyan}{HTML}{60C6C8}
    \definecolor{ansi-cyan-intense}{HTML}{258F8F}
    \definecolor{ansi-white}{HTML}{C5C1B4}
    \definecolor{ansi-white-intense}{HTML}{A1A6B2}

    % commands and environments needed by pandoc snippets
    % extracted from the output of `pandoc -s`
    \providecommand{\tightlist}{%
      \setlength{\itemsep}{0pt}\setlength{\parskip}{0pt}}
    \DefineVerbatimEnvironment{Highlighting}{Verbatim}{commandchars=\\\{\}}
    % Add ',fontsize=\small' for more characters per line
    \newenvironment{Shaded}{}{}
    \newcommand{\KeywordTok}[1]{\textcolor[rgb]{0.00,0.44,0.13}{\textbf{{#1}}}}
    \newcommand{\DataTypeTok}[1]{\textcolor[rgb]{0.56,0.13,0.00}{{#1}}}
    \newcommand{\DecValTok}[1]{\textcolor[rgb]{0.25,0.63,0.44}{{#1}}}
    \newcommand{\BaseNTok}[1]{\textcolor[rgb]{0.25,0.63,0.44}{{#1}}}
    \newcommand{\FloatTok}[1]{\textcolor[rgb]{0.25,0.63,0.44}{{#1}}}
    \newcommand{\CharTok}[1]{\textcolor[rgb]{0.25,0.44,0.63}{{#1}}}
    \newcommand{\StringTok}[1]{\textcolor[rgb]{0.25,0.44,0.63}{{#1}}}
    \newcommand{\CommentTok}[1]{\textcolor[rgb]{0.38,0.63,0.69}{\textit{{#1}}}}
    \newcommand{\OtherTok}[1]{\textcolor[rgb]{0.00,0.44,0.13}{{#1}}}
    \newcommand{\AlertTok}[1]{\textcolor[rgb]{1.00,0.00,0.00}{\textbf{{#1}}}}
    \newcommand{\FunctionTok}[1]{\textcolor[rgb]{0.02,0.16,0.49}{{#1}}}
    \newcommand{\RegionMarkerTok}[1]{{#1}}
    \newcommand{\ErrorTok}[1]{\textcolor[rgb]{1.00,0.00,0.00}{\textbf{{#1}}}}
    \newcommand{\NormalTok}[1]{{#1}}
    
    % Additional commands for more recent versions of Pandoc
    \newcommand{\ConstantTok}[1]{\textcolor[rgb]{0.53,0.00,0.00}{{#1}}}
    \newcommand{\SpecialCharTok}[1]{\textcolor[rgb]{0.25,0.44,0.63}{{#1}}}
    \newcommand{\VerbatimStringTok}[1]{\textcolor[rgb]{0.25,0.44,0.63}{{#1}}}
    \newcommand{\SpecialStringTok}[1]{\textcolor[rgb]{0.73,0.40,0.53}{{#1}}}
    \newcommand{\ImportTok}[1]{{#1}}
    \newcommand{\DocumentationTok}[1]{\textcolor[rgb]{0.73,0.13,0.13}{\textit{{#1}}}}
    \newcommand{\AnnotationTok}[1]{\textcolor[rgb]{0.38,0.63,0.69}{\textbf{\textit{{#1}}}}}
    \newcommand{\CommentVarTok}[1]{\textcolor[rgb]{0.38,0.63,0.69}{\textbf{\textit{{#1}}}}}
    \newcommand{\VariableTok}[1]{\textcolor[rgb]{0.10,0.09,0.49}{{#1}}}
    \newcommand{\ControlFlowTok}[1]{\textcolor[rgb]{0.00,0.44,0.13}{\textbf{{#1}}}}
    \newcommand{\OperatorTok}[1]{\textcolor[rgb]{0.40,0.40,0.40}{{#1}}}
    \newcommand{\BuiltInTok}[1]{{#1}}
    \newcommand{\ExtensionTok}[1]{{#1}}
    \newcommand{\PreprocessorTok}[1]{\textcolor[rgb]{0.74,0.48,0.00}{{#1}}}
    \newcommand{\AttributeTok}[1]{\textcolor[rgb]{0.49,0.56,0.16}{{#1}}}
    \newcommand{\InformationTok}[1]{\textcolor[rgb]{0.38,0.63,0.69}{\textbf{\textit{{#1}}}}}
    \newcommand{\WarningTok}[1]{\textcolor[rgb]{0.38,0.63,0.69}{\textbf{\textit{{#1}}}}}
    
    
    % Define a nice break command that doesn't care if a line doesn't already
    % exist.
    \def\br{\hspace*{\fill} \\* }
    % Math Jax compatability definitions
    \def\gt{>}
    \def\lt{<}
    % Document parameters
    \title{2}
    
    
    

    % Pygments definitions
    
\makeatletter
\def\PY@reset{\let\PY@it=\relax \let\PY@bf=\relax%
    \let\PY@ul=\relax \let\PY@tc=\relax%
    \let\PY@bc=\relax \let\PY@ff=\relax}
\def\PY@tok#1{\csname PY@tok@#1\endcsname}
\def\PY@toks#1+{\ifx\relax#1\empty\else%
    \PY@tok{#1}\expandafter\PY@toks\fi}
\def\PY@do#1{\PY@bc{\PY@tc{\PY@ul{%
    \PY@it{\PY@bf{\PY@ff{#1}}}}}}}
\def\PY#1#2{\PY@reset\PY@toks#1+\relax+\PY@do{#2}}

\expandafter\def\csname PY@tok@w\endcsname{\def\PY@tc##1{\textcolor[rgb]{0.73,0.73,0.73}{##1}}}
\expandafter\def\csname PY@tok@c\endcsname{\let\PY@it=\textit\def\PY@tc##1{\textcolor[rgb]{0.25,0.50,0.50}{##1}}}
\expandafter\def\csname PY@tok@cp\endcsname{\def\PY@tc##1{\textcolor[rgb]{0.74,0.48,0.00}{##1}}}
\expandafter\def\csname PY@tok@k\endcsname{\let\PY@bf=\textbf\def\PY@tc##1{\textcolor[rgb]{0.00,0.50,0.00}{##1}}}
\expandafter\def\csname PY@tok@kp\endcsname{\def\PY@tc##1{\textcolor[rgb]{0.00,0.50,0.00}{##1}}}
\expandafter\def\csname PY@tok@kt\endcsname{\def\PY@tc##1{\textcolor[rgb]{0.69,0.00,0.25}{##1}}}
\expandafter\def\csname PY@tok@o\endcsname{\def\PY@tc##1{\textcolor[rgb]{0.40,0.40,0.40}{##1}}}
\expandafter\def\csname PY@tok@ow\endcsname{\let\PY@bf=\textbf\def\PY@tc##1{\textcolor[rgb]{0.67,0.13,1.00}{##1}}}
\expandafter\def\csname PY@tok@nb\endcsname{\def\PY@tc##1{\textcolor[rgb]{0.00,0.50,0.00}{##1}}}
\expandafter\def\csname PY@tok@nf\endcsname{\def\PY@tc##1{\textcolor[rgb]{0.00,0.00,1.00}{##1}}}
\expandafter\def\csname PY@tok@nc\endcsname{\let\PY@bf=\textbf\def\PY@tc##1{\textcolor[rgb]{0.00,0.00,1.00}{##1}}}
\expandafter\def\csname PY@tok@nn\endcsname{\let\PY@bf=\textbf\def\PY@tc##1{\textcolor[rgb]{0.00,0.00,1.00}{##1}}}
\expandafter\def\csname PY@tok@ne\endcsname{\let\PY@bf=\textbf\def\PY@tc##1{\textcolor[rgb]{0.82,0.25,0.23}{##1}}}
\expandafter\def\csname PY@tok@nv\endcsname{\def\PY@tc##1{\textcolor[rgb]{0.10,0.09,0.49}{##1}}}
\expandafter\def\csname PY@tok@no\endcsname{\def\PY@tc##1{\textcolor[rgb]{0.53,0.00,0.00}{##1}}}
\expandafter\def\csname PY@tok@nl\endcsname{\def\PY@tc##1{\textcolor[rgb]{0.63,0.63,0.00}{##1}}}
\expandafter\def\csname PY@tok@ni\endcsname{\let\PY@bf=\textbf\def\PY@tc##1{\textcolor[rgb]{0.60,0.60,0.60}{##1}}}
\expandafter\def\csname PY@tok@na\endcsname{\def\PY@tc##1{\textcolor[rgb]{0.49,0.56,0.16}{##1}}}
\expandafter\def\csname PY@tok@nt\endcsname{\let\PY@bf=\textbf\def\PY@tc##1{\textcolor[rgb]{0.00,0.50,0.00}{##1}}}
\expandafter\def\csname PY@tok@nd\endcsname{\def\PY@tc##1{\textcolor[rgb]{0.67,0.13,1.00}{##1}}}
\expandafter\def\csname PY@tok@s\endcsname{\def\PY@tc##1{\textcolor[rgb]{0.73,0.13,0.13}{##1}}}
\expandafter\def\csname PY@tok@sd\endcsname{\let\PY@it=\textit\def\PY@tc##1{\textcolor[rgb]{0.73,0.13,0.13}{##1}}}
\expandafter\def\csname PY@tok@si\endcsname{\let\PY@bf=\textbf\def\PY@tc##1{\textcolor[rgb]{0.73,0.40,0.53}{##1}}}
\expandafter\def\csname PY@tok@se\endcsname{\let\PY@bf=\textbf\def\PY@tc##1{\textcolor[rgb]{0.73,0.40,0.13}{##1}}}
\expandafter\def\csname PY@tok@sr\endcsname{\def\PY@tc##1{\textcolor[rgb]{0.73,0.40,0.53}{##1}}}
\expandafter\def\csname PY@tok@ss\endcsname{\def\PY@tc##1{\textcolor[rgb]{0.10,0.09,0.49}{##1}}}
\expandafter\def\csname PY@tok@sx\endcsname{\def\PY@tc##1{\textcolor[rgb]{0.00,0.50,0.00}{##1}}}
\expandafter\def\csname PY@tok@m\endcsname{\def\PY@tc##1{\textcolor[rgb]{0.40,0.40,0.40}{##1}}}
\expandafter\def\csname PY@tok@gh\endcsname{\let\PY@bf=\textbf\def\PY@tc##1{\textcolor[rgb]{0.00,0.00,0.50}{##1}}}
\expandafter\def\csname PY@tok@gu\endcsname{\let\PY@bf=\textbf\def\PY@tc##1{\textcolor[rgb]{0.50,0.00,0.50}{##1}}}
\expandafter\def\csname PY@tok@gd\endcsname{\def\PY@tc##1{\textcolor[rgb]{0.63,0.00,0.00}{##1}}}
\expandafter\def\csname PY@tok@gi\endcsname{\def\PY@tc##1{\textcolor[rgb]{0.00,0.63,0.00}{##1}}}
\expandafter\def\csname PY@tok@gr\endcsname{\def\PY@tc##1{\textcolor[rgb]{1.00,0.00,0.00}{##1}}}
\expandafter\def\csname PY@tok@ge\endcsname{\let\PY@it=\textit}
\expandafter\def\csname PY@tok@gs\endcsname{\let\PY@bf=\textbf}
\expandafter\def\csname PY@tok@gp\endcsname{\let\PY@bf=\textbf\def\PY@tc##1{\textcolor[rgb]{0.00,0.00,0.50}{##1}}}
\expandafter\def\csname PY@tok@go\endcsname{\def\PY@tc##1{\textcolor[rgb]{0.53,0.53,0.53}{##1}}}
\expandafter\def\csname PY@tok@gt\endcsname{\def\PY@tc##1{\textcolor[rgb]{0.00,0.27,0.87}{##1}}}
\expandafter\def\csname PY@tok@err\endcsname{\def\PY@bc##1{\setlength{\fboxsep}{0pt}\fcolorbox[rgb]{1.00,0.00,0.00}{1,1,1}{\strut ##1}}}
\expandafter\def\csname PY@tok@kc\endcsname{\let\PY@bf=\textbf\def\PY@tc##1{\textcolor[rgb]{0.00,0.50,0.00}{##1}}}
\expandafter\def\csname PY@tok@kd\endcsname{\let\PY@bf=\textbf\def\PY@tc##1{\textcolor[rgb]{0.00,0.50,0.00}{##1}}}
\expandafter\def\csname PY@tok@kn\endcsname{\let\PY@bf=\textbf\def\PY@tc##1{\textcolor[rgb]{0.00,0.50,0.00}{##1}}}
\expandafter\def\csname PY@tok@kr\endcsname{\let\PY@bf=\textbf\def\PY@tc##1{\textcolor[rgb]{0.00,0.50,0.00}{##1}}}
\expandafter\def\csname PY@tok@bp\endcsname{\def\PY@tc##1{\textcolor[rgb]{0.00,0.50,0.00}{##1}}}
\expandafter\def\csname PY@tok@fm\endcsname{\def\PY@tc##1{\textcolor[rgb]{0.00,0.00,1.00}{##1}}}
\expandafter\def\csname PY@tok@vc\endcsname{\def\PY@tc##1{\textcolor[rgb]{0.10,0.09,0.49}{##1}}}
\expandafter\def\csname PY@tok@vg\endcsname{\def\PY@tc##1{\textcolor[rgb]{0.10,0.09,0.49}{##1}}}
\expandafter\def\csname PY@tok@vi\endcsname{\def\PY@tc##1{\textcolor[rgb]{0.10,0.09,0.49}{##1}}}
\expandafter\def\csname PY@tok@vm\endcsname{\def\PY@tc##1{\textcolor[rgb]{0.10,0.09,0.49}{##1}}}
\expandafter\def\csname PY@tok@sa\endcsname{\def\PY@tc##1{\textcolor[rgb]{0.73,0.13,0.13}{##1}}}
\expandafter\def\csname PY@tok@sb\endcsname{\def\PY@tc##1{\textcolor[rgb]{0.73,0.13,0.13}{##1}}}
\expandafter\def\csname PY@tok@sc\endcsname{\def\PY@tc##1{\textcolor[rgb]{0.73,0.13,0.13}{##1}}}
\expandafter\def\csname PY@tok@dl\endcsname{\def\PY@tc##1{\textcolor[rgb]{0.73,0.13,0.13}{##1}}}
\expandafter\def\csname PY@tok@s2\endcsname{\def\PY@tc##1{\textcolor[rgb]{0.73,0.13,0.13}{##1}}}
\expandafter\def\csname PY@tok@sh\endcsname{\def\PY@tc##1{\textcolor[rgb]{0.73,0.13,0.13}{##1}}}
\expandafter\def\csname PY@tok@s1\endcsname{\def\PY@tc##1{\textcolor[rgb]{0.73,0.13,0.13}{##1}}}
\expandafter\def\csname PY@tok@mb\endcsname{\def\PY@tc##1{\textcolor[rgb]{0.40,0.40,0.40}{##1}}}
\expandafter\def\csname PY@tok@mf\endcsname{\def\PY@tc##1{\textcolor[rgb]{0.40,0.40,0.40}{##1}}}
\expandafter\def\csname PY@tok@mh\endcsname{\def\PY@tc##1{\textcolor[rgb]{0.40,0.40,0.40}{##1}}}
\expandafter\def\csname PY@tok@mi\endcsname{\def\PY@tc##1{\textcolor[rgb]{0.40,0.40,0.40}{##1}}}
\expandafter\def\csname PY@tok@il\endcsname{\def\PY@tc##1{\textcolor[rgb]{0.40,0.40,0.40}{##1}}}
\expandafter\def\csname PY@tok@mo\endcsname{\def\PY@tc##1{\textcolor[rgb]{0.40,0.40,0.40}{##1}}}
\expandafter\def\csname PY@tok@ch\endcsname{\let\PY@it=\textit\def\PY@tc##1{\textcolor[rgb]{0.25,0.50,0.50}{##1}}}
\expandafter\def\csname PY@tok@cm\endcsname{\let\PY@it=\textit\def\PY@tc##1{\textcolor[rgb]{0.25,0.50,0.50}{##1}}}
\expandafter\def\csname PY@tok@cpf\endcsname{\let\PY@it=\textit\def\PY@tc##1{\textcolor[rgb]{0.25,0.50,0.50}{##1}}}
\expandafter\def\csname PY@tok@c1\endcsname{\let\PY@it=\textit\def\PY@tc##1{\textcolor[rgb]{0.25,0.50,0.50}{##1}}}
\expandafter\def\csname PY@tok@cs\endcsname{\let\PY@it=\textit\def\PY@tc##1{\textcolor[rgb]{0.25,0.50,0.50}{##1}}}

\def\PYZbs{\char`\\}
\def\PYZus{\char`\_}
\def\PYZob{\char`\{}
\def\PYZcb{\char`\}}
\def\PYZca{\char`\^}
\def\PYZam{\char`\&}
\def\PYZlt{\char`\<}
\def\PYZgt{\char`\>}
\def\PYZsh{\char`\#}
\def\PYZpc{\char`\%}
\def\PYZdl{\char`\$}
\def\PYZhy{\char`\-}
\def\PYZsq{\char`\'}
\def\PYZdq{\char`\"}
\def\PYZti{\char`\~}
% for compatibility with earlier versions
\def\PYZat{@}
\def\PYZlb{[}
\def\PYZrb{]}
\makeatother


    % Exact colors from NB
    \definecolor{incolor}{rgb}{0.0, 0.0, 0.5}
    \definecolor{outcolor}{rgb}{0.545, 0.0, 0.0}



    
    % Prevent overflowing lines due to hard-to-break entities
    \sloppy 
    % Setup hyperref package
    \hypersetup{
      breaklinks=true,  % so long urls are correctly broken across lines
      colorlinks=true,
      urlcolor=urlcolor,
      linkcolor=linkcolor,
      citecolor=citecolor,
      }
    % Slightly bigger margins than the latex defaults
    
    \geometry{verbose,tmargin=1in,bmargin=1in,lmargin=1in,rmargin=1in}
    
    

    \begin{document}
    
    
    \maketitle
    
    

    
    \section{Introductory Programming in
R}\label{introductory-programming-in-r}

By Asef Nazari

asef.nazari@monash.edu

Faculty of IT

Monash university

    \section{2. Data Types}\label{data-types}

    \subsection{2.1 Main Data Classes}\label{main-data-classes}

R has five basic or ``atomic'' {classes} of objects:

\begin{itemize}
\tightlist
\item
  numeric:
\item
  double (real numbers): values like 2.3, 3.14, -5.7634 , ...
\item
  integer: values like 0,1,2, -4, ...
\item
  character: values like "GDDS", 'exe'
\item
  logical: TRUE and FALSE (always capital letters)
\item
  complex: we have nothing to do with it in this unit.
\end{itemize}

    \begin{Verbatim}[commandchars=\\\{\}]
{\color{incolor}In [{\color{incolor}3}]:} \PY{k+kp}{typeof}\PY{p}{(}\PY{l+m}{2}\PY{p}{)} \PY{c+c1}{\PYZsh{} numbers by default are double}
        \PY{k+kp}{typeof}\PY{p}{(}\PY{l+m}{2L}\PY{p}{)} \PY{c+c1}{\PYZsh{} to force to be integer}
        \PY{k+kp}{typeof}\PY{p}{(}\PY{l+m}{3.14}\PY{p}{)}
        \PY{k+kp}{typeof}\PY{p}{(}\PY{n+nb+bp}{T}\PY{p}{)}
        \PY{k+kp}{typeof}\PY{p}{(}\PY{k+kc}{TRUE}\PY{p}{)}
        \PY{k+kp}{typeof}\PY{p}{(}\PY{l+s}{\PYZdq{}}\PY{l+s}{TRUE\PYZdq{}}\PY{p}{)}
        \PY{k+kp}{typeof}\PY{p}{(}\PY{l+m}{3}\PY{l+m}{+2i}\PY{p}{)}
\end{Verbatim}


    'double'

    
    'integer'

    
    'double'

    
    'logical'

    
    'logical'

    
    'character'

    
    'complex'

    
    \subsection{2.2 Vectors}\label{vectors}

The most basic {type} of R objects is a vector. All the objects we used
so far are vectors of length 1. Vectors are variables with one or more
values of the same type, e.g., all are of numeric class. For example, a
numeric vector might consist of the numbers (1.2, 2.3, 0.2, 1.1).

\begin{itemize}
\tightlist
\item
  Vectors are created by c() function (concatenatation function)
\item
  Also, they ca be created by vector() function: v \textless{}-
  vector("numeric", length=5)
\item
  should contain objects of the same class
\item
  if you put objects from different classes, an implicit coercion (the
  calss of value would be changed) will happen
\item
  Creating variables using seq and rep functions.
\end{itemize}

    \begin{Verbatim}[commandchars=\\\{\}]
{\color{incolor}In [{\color{incolor}1}]:} v1 \PY{o}{\PYZlt{}\PYZhy{}} \PY{k+kt}{c}\PY{p}{(}\PY{l+m}{5}\PY{p}{,}\PY{l+m}{7}\PY{p}{,}\PY{l+m}{9}\PY{p}{)} \PY{c+c1}{\PYZsh{} a vector called v1 is created.}
\end{Verbatim}


    \begin{Verbatim}[commandchars=\\\{\}]
{\color{incolor}In [{\color{incolor}2}]:} v1
\end{Verbatim}


    \begin{enumerate*}
\item 5
\item 7
\item 9
\end{enumerate*}


    
    \begin{Verbatim}[commandchars=\\\{\}]
{\color{incolor}In [{\color{incolor}4}]:} \PY{k+kp}{print}\PY{p}{(}v1\PY{p}{)}
        \PY{c+c1}{\PYZsh{}this says v1 is a vector, or a sequence of objects, and the first one is 5.}
\end{Verbatim}


    \begin{Verbatim}[commandchars=\\\{\}]
[1] 5 7 9

    \end{Verbatim}

    \begin{Verbatim}[commandchars=\\\{\}]
{\color{incolor}In [{\color{incolor}5}]:} v2 \PY{o}{\PYZlt{}\PYZhy{}} \PY{l+m}{3}\PY{o}{:}\PY{l+m}{35} \PY{c+c1}{\PYZsh{} a sequence of consecutive integers are put in v2. The sequence starts from 3 and goes to 35}
        \PY{k+kp}{print}\PY{p}{(}v2\PY{p}{)}
        \PY{c+c1}{\PYZsh{} the first item is 3 and the 26th item is 28.}
\end{Verbatim}


    \begin{Verbatim}[commandchars=\\\{\}]
 [1]  3  4  5  6  7  8  9 10 11 12 13 14 15 16 17 18 19 20 21 22 23 24 25 26 27
[26] 28 29 30 31 32 33 34 35

    \end{Verbatim}

    \begin{Verbatim}[commandchars=\\\{\}]
{\color{incolor}In [{\color{incolor}6}]:} v3 \PY{o}{\PYZlt{}\PYZhy{}} \PY{k+kt}{c}\PY{p}{(}\PY{l+s}{\PYZdq{}}\PY{l+s}{Helo\PYZdq{}}\PY{p}{,} \PY{l+s}{\PYZdq{}}\PY{l+s}{Hi\PYZdq{}}\PY{p}{,} \PY{l+s}{\PYZdq{}}\PY{l+s}{Bye\PYZdq{}}\PY{p}{)} \PY{c+c1}{\PYZsh{} a vector of characters}
        \PY{k+kp}{print}\PY{p}{(}v3\PY{p}{)}
\end{Verbatim}


    \begin{Verbatim}[commandchars=\\\{\}]
[1] "Helo" "Hi"   "Bye" 

    \end{Verbatim}

    \begin{Verbatim}[commandchars=\\\{\}]
{\color{incolor}In [{\color{incolor}7}]:} v4 \PY{o}{\PYZlt{}\PYZhy{}} \PY{k+kt}{c}\PY{p}{(}\PY{k+kc}{TRUE}\PY{p}{,} \PY{k+kc}{TRUE}\PY{p}{,} \PY{k+kc}{FALSE}\PY{p}{,} \PY{k+kc}{TRUE}\PY{p}{,} \PY{k+kc}{TRUE}\PY{p}{)} \PY{c+c1}{\PYZsh{} a vector of logical values}
        v4
\end{Verbatim}


    \begin{enumerate*}
\item TRUE
\item TRUE
\item FALSE
\item TRUE
\item TRUE
\end{enumerate*}


    
    \begin{Verbatim}[commandchars=\\\{\}]
{\color{incolor}In [{\color{incolor}8}]:} \PY{k+kp}{length}\PY{p}{(}v4\PY{p}{)} \PY{c+c1}{\PYZsh{} gives the length of a vector}
\end{Verbatim}


    5

    
    \begin{Verbatim}[commandchars=\\\{\}]
{\color{incolor}In [{\color{incolor}9}]:} v5 \PY{o}{\PYZlt{}\PYZhy{}} \PY{k+kp}{seq}\PY{p}{(}\PY{l+m}{2}\PY{p}{,}\PY{l+m}{8}\PY{p}{)} \PY{c+c1}{\PYZsh{}another way of making a vector of consecutive numbers. Same as 2:8}
        v5
\end{Verbatim}


    \begin{enumerate*}
\item 2
\item 3
\item 4
\item 5
\item 6
\item 7
\item 8
\end{enumerate*}


    
    \begin{Verbatim}[commandchars=\\\{\}]
{\color{incolor}In [{\color{incolor}13}]:} v6 \PY{o}{\PYZlt{}\PYZhy{}} \PY{k+kp}{seq}\PY{p}{(}from\PY{o}{=}\PY{l+m}{3}\PY{p}{,} to\PY{o}{=}\PY{l+m}{10}\PY{p}{,} by\PY{o}{=}\PY{l+m}{2}\PY{p}{)} \PY{c+c1}{\PYZsh{} equally you can write seq(3,10,2)}
         \PY{k+kp}{print}\PY{p}{(}v6\PY{p}{)}
\end{Verbatim}


    \begin{Verbatim}[commandchars=\\\{\}]
[1] 3 5 7 9

    \end{Verbatim}

    \begin{Verbatim}[commandchars=\\\{\}]
{\color{incolor}In [{\color{incolor}12}]:} \PY{c+c1}{\PYZsh{}learn more about seq() function by typing ? seq}
         \PY{o}{?}\PY{k+kp}{seq}
\end{Verbatim}


    \begin{Verbatim}[commandchars=\\\{\}]
{\color{incolor}In [{\color{incolor}4}]:} \PY{k+kt}{vector}\PY{p}{(}\PY{l+s}{\PYZdq{}}\PY{l+s}{logical\PYZdq{}}\PY{p}{,} length\PY{o}{=}\PY{l+m}{4}\PY{p}{)}
\end{Verbatim}


    \begin{enumerate*}
\item FALSE
\item FALSE
\item FALSE
\item FALSE
\end{enumerate*}


    
    \begin{Verbatim}[commandchars=\\\{\}]
{\color{incolor}In [{\color{incolor}8}]:} \PY{k+kt}{vector}\PY{p}{(}\PY{l+s}{\PYZdq{}}\PY{l+s}{complex\PYZdq{}}\PY{p}{,} length\PY{o}{=}\PY{l+m}{4}\PY{p}{)}
\end{Verbatim}


    \begin{enumerate*}
\item 0+0i
\item 0+0i
\item 0+0i
\item 0+0i
\end{enumerate*}


    
    \begin{Verbatim}[commandchars=\\\{\}]
{\color{incolor}In [{\color{incolor}14}]:} v7 \PY{o}{\PYZlt{}\PYZhy{}} \PY{k+kt}{vector}\PY{p}{(}mode\PY{o}{=}\PY{l+s}{\PYZdq{}}\PY{l+s}{numeric\PYZdq{}}\PY{p}{,} length\PY{o}{=}\PY{l+m}{5}\PY{p}{)} \PY{c+c1}{\PYZsh{} another way of creating a vector}
         \PY{k+kp}{print}\PY{p}{(}v7\PY{p}{)}
\end{Verbatim}


    \begin{Verbatim}[commandchars=\\\{\}]
[1] 0 0 0 0 0

    \end{Verbatim}

    \begin{Verbatim}[commandchars=\\\{\}]
{\color{incolor}In [{\color{incolor}15}]:} v8 \PY{o}{\PYZlt{}\PYZhy{}} \PY{k+kt}{c}\PY{p}{(}\PY{l+m}{5}\PY{p}{,} \PY{l+s}{\PYZdq{}}\PY{l+s}{a\PYZdq{}}\PY{p}{,} \PY{l+m}{2}\PY{p}{)} \PY{c+c1}{\PYZsh{}different types, so a coercion happens. Be very careful about this.}
         \PY{k+kp}{print}\PY{p}{(}v8\PY{p}{)}
\end{Verbatim}


    \begin{Verbatim}[commandchars=\\\{\}]
[1] "5" "a" "2"

    \end{Verbatim}

    \begin{Verbatim}[commandchars=\\\{\}]
{\color{incolor}In [{\color{incolor}30}]:} \PY{c+c1}{\PYZsh{}accessing elements of a vector}
         v8\PY{p}{[}\PY{l+m}{1}\PY{p}{]}
         \PY{k+kp}{print}\PY{p}{(}v8\PY{p}{[}\PY{l+m}{2}\PY{p}{]}\PY{p}{)}
\end{Verbatim}


    '5'

    
    \begin{Verbatim}[commandchars=\\\{\}]
[1] "a"

    \end{Verbatim}

    \begin{Verbatim}[commandchars=\\\{\}]
{\color{incolor}In [{\color{incolor}3}]:} vv \PY{o}{\PYZlt{}\PYZhy{}} \PY{k+kt}{c}\PY{p}{(}\PY{l+m}{1}\PY{p}{,}\PY{l+m}{2}\PY{p}{,}\PY{l+m}{3}\PY{p}{)}
        vv
        vv\PY{p}{[}\PY{l+m}{2}\PY{p}{]} \PY{c+c1}{\PYZsh{}prints the second item}
        vv\PY{p}{[}\PY{l+m}{2}\PY{p}{]} \PY{o}{\PYZlt{}\PYZhy{}} \PY{l+m}{257} \PY{c+c1}{\PYZsh{} changes the value stored in the second element}
        vv
\end{Verbatim}


    \begin{enumerate*}
\item 1
\item 2
\item 3
\end{enumerate*}


    
    2

    
    \begin{enumerate*}
\item 1
\item 257
\item 3
\end{enumerate*}


    
    \begin{Verbatim}[commandchars=\\\{\}]
{\color{incolor}In [{\color{incolor}46}]:} \PY{c+c1}{\PYZsh{}to choose more than one element from a vector}
         x \PY{o}{\PYZlt{}\PYZhy{}} \PY{k+kt}{c}\PY{p}{(}\PY{l+m}{12.2}\PY{p}{,} \PY{l+m}{52.3}\PY{p}{,} \PY{l+m}{10.2}\PY{p}{,} \PY{l+m}{11.1}\PY{p}{)}
         x\PY{p}{[}\PY{l+m}{1}\PY{p}{]} \PY{c+c1}{\PYZsh{} only the first element}
         x\PY{p}{[}\PY{k+kt}{c}\PY{p}{(}\PY{l+m}{1}\PY{p}{,}\PY{l+m}{3}\PY{p}{)}\PY{p}{]} \PY{c+c1}{\PYZsh{} the first and third elemment}
\end{Verbatim}


    12.2

    
    \begin{enumerate*}
\item 12.2
\item 10.2
\end{enumerate*}


    
    \begin{Verbatim}[commandchars=\\\{\}]
{\color{incolor}In [{\color{incolor}66}]:} \PY{c+c1}{\PYZsh{} Adding an element to the end of a list}
         v \PY{o}{\PYZlt{}\PYZhy{}} \PY{k+kt}{c}\PY{p}{(}\PY{l+m}{1}\PY{p}{,}\PY{l+m}{2}\PY{p}{,}\PY{l+m}{3}\PY{p}{)}
         \PY{k+kp}{print}\PY{p}{(}v\PY{p}{)}
\end{Verbatim}


    \begin{Verbatim}[commandchars=\\\{\}]
[1] 1 2 3

    \end{Verbatim}

    \begin{Verbatim}[commandchars=\\\{\}]
{\color{incolor}In [{\color{incolor}68}]:} v \PY{o}{\PYZlt{}\PYZhy{}} \PY{k+kt}{c}\PY{p}{(}v\PY{p}{,} \PY{l+m}{100}\PY{p}{)} \PY{c+c1}{\PYZsh{} 100 is added to the end of a vector}
         \PY{k+kp}{print}\PY{p}{(}v\PY{p}{)}
\end{Verbatim}


    \begin{Verbatim}[commandchars=\\\{\}]
[1]   1   2   3 100

    \end{Verbatim}

    \begin{Verbatim}[commandchars=\\\{\}]
{\color{incolor}In [{\color{incolor}10}]:} \PY{c+c1}{\PYZsh{} Create sequential data}
         x1 \PY{o}{\PYZlt{}\PYZhy{}} \PY{l+m}{0}\PY{o}{:}\PY{l+m}{10}  \PY{c+c1}{\PYZsh{} Assigns number 0 through 10 to x1}
         x2 \PY{o}{\PYZlt{}\PYZhy{}} \PY{l+m}{10}\PY{o}{:}\PY{l+m}{0}  \PY{c+c1}{\PYZsh{} Assigns number 10 through 0 to x2}
         x3 \PY{o}{\PYZlt{}\PYZhy{}} \PY{k+kp}{seq}\PY{p}{(}\PY{l+m}{10}\PY{p}{)}  \PY{c+c1}{\PYZsh{} Counts from 1 to 10}
         x4 \PY{o}{\PYZlt{}\PYZhy{}} \PY{k+kp}{seq}\PY{p}{(}\PY{l+m}{30}\PY{p}{,} \PY{l+m}{0}\PY{p}{,} by \PY{o}{=} \PY{l+m}{\PYZhy{}3}\PY{p}{)}  \PY{c+c1}{\PYZsh{} Counts down by 3}
\end{Verbatim}


    \begin{Verbatim}[commandchars=\\\{\}]
{\color{incolor}In [{\color{incolor}12}]:} x4
\end{Verbatim}


    \begin{enumerate*}
\item 30
\item 27
\item 24
\item 21
\item 18
\item 15
\item 12
\item 9
\item 6
\item 3
\item 0
\end{enumerate*}


    
    \begin{Verbatim}[commandchars=\\\{\}]
{\color{incolor}In [{\color{incolor}4}]:} x \PY{o}{\PYZlt{}\PYZhy{}} \PY{k+kt}{c}\PY{p}{(}\PY{l+m}{1}\PY{p}{,}\PY{l+m}{3}\PY{p}{,}\PY{l+m}{6}\PY{p}{,}\PY{l+m}{9}\PY{p}{,}\PY{l+m}{0}\PY{p}{)}
        x
        x\PY{p}{[}\PY{l+m}{\PYZhy{}2}\PY{p}{]} \PY{c+c1}{\PYZsh{} all the elements except the second element}
        x\PY{p}{[}\PY{l+m}{3}\PY{p}{]} \PY{o}{\PYZlt{}\PYZhy{}} \PY{l+m}{200} \PY{c+c1}{\PYZsh{}modify an element}
        x
        \PY{c+c1}{\PYZsh{} to delete a vector}
        x \PY{o}{\PYZlt{}\PYZhy{}} \PY{k+kc}{NULL}
        x
\end{Verbatim}


    \begin{enumerate*}
\item 1
\item 3
\item 6
\item 9
\item 0
\end{enumerate*}


    
    \begin{enumerate*}
\item 1
\item 6
\item 9
\item 0
\end{enumerate*}


    
    \begin{enumerate*}
\item 1
\item 3
\item 200
\item 9
\item 0
\end{enumerate*}


    
    
    \begin{verbatim}
NULL
    \end{verbatim}

    
    \begin{Verbatim}[commandchars=\\\{\}]
{\color{incolor}In [{\color{incolor}16}]:} x \PY{o}{\PYZlt{}\PYZhy{}} \PY{k+kt}{c}\PY{p}{(}\PY{l+m}{2}\PY{p}{,} \PY{l+m}{9}\PY{p}{,} \PY{l+m}{7}\PY{p}{)}
         x
         y \PY{o}{\PYZlt{}\PYZhy{}} \PY{k+kt}{c}\PY{p}{(}x\PY{p}{,} x\PY{p}{,} \PY{l+m}{10}\PY{p}{)}
         y
\end{Verbatim}


    \begin{enumerate*}
\item 2
\item 9
\item 7
\end{enumerate*}


    
    \begin{enumerate*}
\item 2
\item 9
\item 7
\item 2
\item 9
\item 7
\item 10
\end{enumerate*}


    
    \begin{Verbatim}[commandchars=\\\{\}]
{\color{incolor}In [{\color{incolor}18}]:} \PY{k+kp}{round}\PY{p}{(}\PY{k+kp}{seq}\PY{p}{(}\PY{l+m}{1}\PY{p}{,}\PY{l+m}{3}\PY{p}{,}length\PY{o}{=}\PY{l+m}{10}\PY{p}{)}\PY{p}{,} \PY{l+m}{2}\PY{p}{)}
\end{Verbatim}


    \begin{enumerate*}
\item 1
\item 1.22
\item 1.44
\item 1.67
\item 1.89
\item 2.11
\item 2.33
\item 2.56
\item 2.78
\item 3
\end{enumerate*}


    
    \begin{Verbatim}[commandchars=\\\{\}]
{\color{incolor}In [{\color{incolor}30}]:} \PY{k+kp}{seq}\PY{p}{(}from \PY{o}{=} \PY{l+m}{2}\PY{p}{,} by \PY{o}{=} \PY{l+m}{\PYZhy{}0.1}\PY{p}{,} length.out \PY{o}{=} \PY{l+m}{4}\PY{p}{)}
\end{Verbatim}


    \begin{enumerate*}
\item 2
\item 1.9
\item 1.8
\item 1.7
\end{enumerate*}


    
    \begin{Verbatim}[commandchars=\\\{\}]
{\color{incolor}In [{\color{incolor}20}]:} x \PY{o}{\PYZlt{}\PYZhy{}} \PY{k+kp}{rep}\PY{p}{(}\PY{l+m}{3}\PY{p}{,}\PY{l+m}{4}\PY{p}{)}
         x
\end{Verbatim}


    \begin{enumerate*}
\item 3
\item 3
\item 3
\item 3
\end{enumerate*}


    
    \begin{Verbatim}[commandchars=\\\{\}]
{\color{incolor}In [{\color{incolor}21}]:} \PY{k+kp}{rep}\PY{p}{(}\PY{l+m}{1}\PY{o}{:}\PY{l+m}{5}\PY{p}{,}\PY{l+m}{3}\PY{p}{)}
\end{Verbatim}


    \begin{enumerate*}
\item 1
\item 2
\item 3
\item 4
\item 5
\item 1
\item 2
\item 3
\item 4
\item 5
\item 1
\item 2
\item 3
\item 4
\item 5
\end{enumerate*}


    
    \begin{Verbatim}[commandchars=\\\{\}]
{\color{incolor}In [{\color{incolor}22}]:} x \PY{o}{\PYZlt{}\PYZhy{}} \PY{k+kt}{c}\PY{p}{(}\PY{l+m}{7}\PY{p}{,}\PY{l+m}{3}\PY{p}{,}\PY{l+m}{5}\PY{p}{,}\PY{l+m}{2}\PY{p}{,}\PY{l+m}{0}\PY{p}{,}\PY{l+m}{1}\PY{p}{)}
         y \PY{o}{\PYZlt{}\PYZhy{}} x\PY{p}{[}\PY{l+m}{\PYZhy{}3}\PY{p}{]}
         y
\end{Verbatim}


    \begin{enumerate*}
\item 7
\item 3
\item 2
\item 0
\item 1
\end{enumerate*}


    
    \begin{Verbatim}[commandchars=\\\{\}]
{\color{incolor}In [{\color{incolor}23}]:} y \PY{o}{\PYZlt{}\PYZhy{}} x\PY{p}{[}\PY{o}{\PYZhy{}}\PY{k+kp}{length}\PY{p}{(}x\PY{p}{)}\PY{p}{]} \PY{c+c1}{\PYZsh{} always delets the final element}
         y
\end{Verbatim}


    \begin{enumerate*}
\item 7
\item 3
\item 5
\item 2
\item 0
\end{enumerate*}


    
    \subsection{2.3 Lists}\label{lists}

Other basic object in R is a list. A list is very similar to a vector,
but it could contain objects from different classes. You can create a
list using list() function. The main functionality of lists in putting
outputs of functions inside. Later we will see an important example of
lm() functions.

    \begin{Verbatim}[commandchars=\\\{\}]
{\color{incolor}In [{\color{incolor}1}]:} L1 \PY{o}{\PYZlt{}\PYZhy{}} \PY{k+kt}{list}\PY{p}{(}\PY{l+m}{5}\PY{p}{,} \PY{l+s}{\PYZdq{}}\PY{l+s}{a\PYZdq{}}\PY{p}{,} \PY{l+m}{2}\PY{p}{)}
        \PY{k+kp}{print}\PY{p}{(}L1\PY{p}{)}
        \PY{c+c1}{\PYZsh{} L1 has 3 elements, and each element is considered as a vector}
        \PY{c+c1}{\PYZsh{}pay attention to double brackets. It shows the elements of the list}
\end{Verbatim}


    \begin{Verbatim}[commandchars=\\\{\}]
[[1]]
[1] 5

[[2]]
[1] "a"

[[3]]
[1] 2


    \end{Verbatim}

    \begin{Verbatim}[commandchars=\\\{\}]
{\color{incolor}In [{\color{incolor}2}]:} L1 \PY{c+c1}{\PYZsh{}auto printing}
\end{Verbatim}


    \begin{enumerate}
\item 5
\item 'a'
\item 2
\end{enumerate}


    
    \begin{Verbatim}[commandchars=\\\{\}]
{\color{incolor}In [{\color{incolor}18}]:} \PY{k+kp}{length}\PY{p}{(}L1\PY{p}{)}
\end{Verbatim}


    3

    
    \begin{Verbatim}[commandchars=\\\{\}]
{\color{incolor}In [{\color{incolor}4}]:} L2 \PY{o}{\PYZlt{}\PYZhy{}} \PY{k+kt}{list}\PY{p}{(}\PY{k+kt}{c}\PY{p}{(}\PY{l+m}{1}\PY{p}{,}\PY{l+m}{2}\PY{p}{,}\PY{l+m}{3}\PY{p}{)}\PY{p}{,} \PY{k+kt}{c}\PY{p}{(}\PY{l+s}{\PYZdq{}}\PY{l+s}{One\PYZdq{}}\PY{p}{,} \PY{l+s}{\PYZdq{}}\PY{l+s}{Two\PYZdq{}}\PY{p}{)}\PY{p}{,} \PY{k+kc}{TRUE}\PY{p}{)}
        \PY{k+kp}{print}\PY{p}{(}L2\PY{p}{)}
\end{Verbatim}


    \begin{Verbatim}[commandchars=\\\{\}]
[[1]]
[1] 1 2 3

[[2]]
[1] "One" "Two"

[[3]]
[1] TRUE


    \end{Verbatim}

    \begin{Verbatim}[commandchars=\\\{\}]
{\color{incolor}In [{\color{incolor}6}]:} L2
\end{Verbatim}


    \begin{enumerate}
\item \begin{enumerate*}
\item 1
\item 2
\item 3
\end{enumerate*}

\item \begin{enumerate*}
\item 'One'
\item 'Two'
\end{enumerate*}

\item TRUE
\end{enumerate}


    
    \begin{Verbatim}[commandchars=\\\{\}]
{\color{incolor}In [{\color{incolor}31}]:} L1\PY{p}{[}\PY{l+m}{1}\PY{p}{]}
\end{Verbatim}


    \begin{enumerate}
\item 5
\end{enumerate}


    
    \begin{Verbatim}[commandchars=\\\{\}]
{\color{incolor}In [{\color{incolor}33}]:} \PY{k+kp}{print}\PY{p}{(}L2\PY{p}{[}\PY{l+m}{1}\PY{p}{]}\PY{p}{)}
\end{Verbatim}


    \begin{Verbatim}[commandchars=\\\{\}]
[[1]]
[1] 1 2 3


    \end{Verbatim}

    \begin{Verbatim}[commandchars=\\\{\}]
{\color{incolor}In [{\color{incolor}34}]:} \PY{k+kp}{print}\PY{p}{(}L2\PY{p}{[[}\PY{l+m}{1}\PY{p}{]]}\PY{p}{)}
\end{Verbatim}


    \begin{Verbatim}[commandchars=\\\{\}]
[1] 1 2 3

    \end{Verbatim}

    \subsection{2.4 Numbers}\label{numbers}

Numbers in R are considerd as \textbf{numeric}, (as real numbers with
double precision) . If you want an integer, you need to explicitly add
\textbf{L} to the end of the number, otherwise it is a double.

Special numbers: * \textbf{Inf}, infinity, for \(\frac{1}{0}\) *
\textbf{NaN}, not a number, for \(\frac{0}{0}\) * \textbf{NA} can be
thought as a \textbf{missing value}

    \begin{Verbatim}[commandchars=\\\{\}]
{\color{incolor}In [{\color{incolor}31}]:} x \PY{o}{\PYZlt{}\PYZhy{}} \PY{l+m}{1}
         \PY{k+kp}{print}\PY{p}{(}x\PY{p}{)}
         \PY{k+kp}{class}\PY{p}{(}x\PY{p}{)}
         \PY{k+kp}{typeof}\PY{p}{(}x\PY{p}{)}
         y \PY{o}{\PYZlt{}\PYZhy{}} \PY{l+m}{1L}
         \PY{k+kp}{print}\PY{p}{(}y\PY{p}{)}
         \PY{k+kp}{class}\PY{p}{(}y\PY{p}{)}
         \PY{k+kp}{typeof}\PY{p}{(}y\PY{p}{)}
\end{Verbatim}


    \begin{Verbatim}[commandchars=\\\{\}]
[1] 1

    \end{Verbatim}

    'numeric'

    
    'double'

    
    \begin{Verbatim}[commandchars=\\\{\}]
[1] 1

    \end{Verbatim}

    'integer'

    
    'integer'

    
    \begin{Verbatim}[commandchars=\\\{\}]
{\color{incolor}In [{\color{incolor}50}]:} c1 \PY{o}{\PYZlt{}\PYZhy{}} \PY{l+s}{\PYZdq{}}\PY{l+s}{Heloo\PYZdq{}} \PY{c+c1}{\PYZsh{} character variable}
         c2 \PY{o}{\PYZlt{}\PYZhy{}} \PY{l+s}{\PYZdq{}}\PY{l+s}{The World!\PYZdq{}} \PY{c+c1}{\PYZsh{} another character variable}
         \PY{k+kp}{paste}\PY{p}{(}c1\PY{p}{,} c2\PY{p}{)}
         \PY{k+kp}{print}\PY{p}{(}\PY{k+kt}{c}\PY{p}{(}c1\PY{p}{,} c2\PY{p}{)}\PY{p}{)}
\end{Verbatim}


    'Heloo The World!'

    
    \begin{Verbatim}[commandchars=\\\{\}]
[1] "Heloo"      "The World!"

    \end{Verbatim}

    \begin{Verbatim}[commandchars=\\\{\}]
{\color{incolor}In [{\color{incolor}29}]:} \PY{k+kp}{sqrt}\PY{p}{(}\PY{l+m}{\PYZhy{}2}\PY{p}{)}  \PY{c+c1}{\PYZsh{}NaN stands for “not a number”}
\end{Verbatim}


    \begin{Verbatim}[commandchars=\\\{\}]
Warning message:
In sqrt(-2): NaNs produced
    \end{Verbatim}

    NaN

    
    \subsection{2.5 Changing Class of a
Value}\label{changing-class-of-a-value}

You saw that a vector contains values of only one class. If different
classes mixed together by having valuesw ith different classes in a
vector, an implicit coercion happens. It means R will convert all the
values to a class that are the same. However, sometimes we want to
change the type of a value ourselves, so we implemenet an explicit
coercion by as.SomeClass() functions. * as.numeric() to change the type
into numeric if it is possibel * as.logical() to change into logical if
it is possible * as.character() * as.complex() * as.integer()

Sometimes R cannot convert one type to another, and gives NA. Also, you
will get warning from R.

    \begin{Verbatim}[commandchars=\\\{\}]
{\color{incolor}In [{\color{incolor}21}]:} x \PY{o}{\PYZlt{}\PYZhy{}} \PY{l+m}{1}\PY{o}{:}\PY{l+m}{5} \PY{c+c1}{\PYZsh{}sequence of numbers}
         \PY{k+kp}{class}\PY{p}{(}x\PY{p}{)}
         y \PY{o}{\PYZlt{}\PYZhy{}} \PY{k+kp}{as.numeric}\PY{p}{(}x\PY{p}{)}
         \PY{k+kp}{class}\PY{p}{(}y\PY{p}{)}
         z \PY{o}{\PYZlt{}\PYZhy{}} \PY{k+kp}{as.logical}\PY{p}{(}x\PY{p}{)}
         \PY{k+kp}{print}\PY{p}{(}z\PY{p}{)}
         \PY{k+kp}{class}\PY{p}{(}z\PY{p}{)}
         u \PY{o}{\PYZlt{}\PYZhy{}} \PY{k+kp}{as.character}\PY{p}{(}z\PY{p}{)}
         \PY{k+kp}{print}\PY{p}{(}u\PY{p}{)}
         \PY{k+kp}{class}\PY{p}{(}u\PY{p}{)}
\end{Verbatim}


    'integer'

    
    'numeric'

    
    \begin{Verbatim}[commandchars=\\\{\}]
[1] TRUE TRUE TRUE TRUE TRUE

    \end{Verbatim}

    'logical'

    
    \begin{Verbatim}[commandchars=\\\{\}]
[1] "TRUE" "TRUE" "TRUE" "TRUE" "TRUE"

    \end{Verbatim}

    'character'

    
    \begin{Verbatim}[commandchars=\\\{\}]
{\color{incolor}In [{\color{incolor}22}]:} t \PY{o}{\PYZlt{}\PYZhy{}} \PY{k+kp}{as.numeric}\PY{p}{(}u\PY{p}{)}
         \PY{k+kp}{t}
         \PY{k+kp}{class}\PY{p}{(}\PY{k+kp}{t}\PY{p}{)}
\end{Verbatim}


    \begin{Verbatim}[commandchars=\\\{\}]
Warning message:
In eval(expr, envir, enclos): NAs introduced by coercion
    \end{Verbatim}

    \begin{enumerate*}
\item NA
\item NA
\item NA
\item NA
\item NA
\end{enumerate*}


    
    'numeric'

    
    \begin{Verbatim}[commandchars=\\\{\}]
{\color{incolor}In [{\color{incolor}28}]:} \PY{c+c1}{\PYZsh{}list does not have any problm with mixing data types. Very poerful!}
         x \PY{o}{\PYZlt{}\PYZhy{}} \PY{k+kt}{list}\PY{p}{(}\PY{l+m}{14}\PY{p}{,} \PY{l+s}{\PYZdq{}}\PY{l+s}{Hello\PYZdq{}}\PY{p}{,} \PY{k+kc}{TRUE}\PY{p}{,} \PY{k+kt}{list}\PY{p}{(}\PY{l+m}{23}\PY{p}{,} \PY{l+s}{\PYZdq{}}\PY{l+s}{Hi\PYZdq{}}\PY{p}{,} \PY{k+kc}{TRUE}\PY{p}{,} \PY{k+kc}{FALSE}\PY{p}{)}\PY{p}{)}
         x
\end{Verbatim}


    \begin{enumerate}
\item 14
\item 'Hello'
\item TRUE
\item \begin{enumerate}
\item 23
\item 'Hi'
\item TRUE
\item FALSE
\end{enumerate}

\end{enumerate}


    
    \begin{Verbatim}[commandchars=\\\{\}]
{\color{incolor}In [{\color{incolor}29}]:} \PY{k+kp}{print}\PY{p}{(}x\PY{p}{)}
         \PY{c+c1}{\PYZsh{}elements of list has double brackets around them. Other objects have single bracket}
\end{Verbatim}


    \begin{Verbatim}[commandchars=\\\{\}]
[[1]]
[1] 14

[[2]]
[1] "Hello"

[[3]]
[1] TRUE

[[4]]
[[4]][[1]]
[1] 23

[[4]][[2]]
[1] "Hi"

[[4]][[3]]
[1] TRUE

[[4]][[4]]
[1] FALSE



    \end{Verbatim}

    \subsection{2.6 Factors}\label{factors}

Categorical data in R are represented using factors. We will learn a lot
about this type of data soon. Factors are stored as integers, but they
are assigned labels. R sorts factors in alphabetical oredr. Factors can
be ordered or unordered. R considers factors as nominal categorical
variables, and "ordered" as ordinal categorical variables.

    \begin{Verbatim}[commandchars=\\\{\}]
{\color{incolor}In [{\color{incolor}7}]:} x \PY{o}{\PYZlt{}\PYZhy{}} \PY{k+kp}{factor}\PY{p}{(}\PY{k+kt}{c}\PY{p}{(}\PY{l+s}{\PYZdq{}}\PY{l+s}{male\PYZdq{}}\PY{p}{,} \PY{l+s}{\PYZdq{}}\PY{l+s}{fmale\PYZdq{}}\PY{p}{,} \PY{l+s}{\PYZdq{}}\PY{l+s}{male\PYZdq{}}\PY{p}{,} \PY{l+s}{\PYZdq{}}\PY{l+s}{male\PYZdq{}}\PY{p}{,} \PY{l+s}{\PYZdq{}}\PY{l+s}{fmale\PYZdq{}}\PY{p}{,} \PY{l+s}{\PYZdq{}}\PY{l+s}{male\PYZdq{}}\PY{p}{)}\PY{p}{)} \PY{c+c1}{\PYZsh{}create a factor object}
        \PY{k+kp}{print}\PY{p}{(}x\PY{p}{)}
\end{Verbatim}


    \begin{Verbatim}[commandchars=\\\{\}]
[1] male  fmale male  male  fmale male 
Levels: fmale male

    \end{Verbatim}

    \begin{Verbatim}[commandchars=\\\{\}]
{\color{incolor}In [{\color{incolor}9}]:} \PY{k+kp}{levels}\PY{p}{(}x\PY{p}{)} \PY{c+c1}{\PYZsh{}alphabetical order}
\end{Verbatim}


    \begin{enumerate*}
\item 'fmale'
\item 'male'
\end{enumerate*}


    
    \begin{Verbatim}[commandchars=\\\{\}]
{\color{incolor}In [{\color{incolor}10}]:} \PY{k+kp}{nlevels}\PY{p}{(}x\PY{p}{)}
\end{Verbatim}


    2

    
    \begin{Verbatim}[commandchars=\\\{\}]
{\color{incolor}In [{\color{incolor}11}]:} \PY{k+kp}{unclass}\PY{p}{(}x\PY{p}{)}
\end{Verbatim}


    \begin{enumerate*}
\item 2
\item 1
\item 2
\item 2
\item 1
\item 2
\end{enumerate*}


    
    \begin{Verbatim}[commandchars=\\\{\}]
{\color{incolor}In [{\color{incolor}37}]:} \PY{k+kp}{table}\PY{p}{(}x\PY{p}{)} \PY{c+c1}{\PYZsh{}gives frequency count}
\end{Verbatim}


    
    \begin{verbatim}
x
fmale  male 
    2     4 
    \end{verbatim}

    
    \begin{Verbatim}[commandchars=\\\{\}]
{\color{incolor}In [{\color{incolor}39}]:} \PY{k+kp}{levels}\PY{p}{(}x\PY{p}{)}
\end{Verbatim}


    \begin{enumerate*}
\item 'fmale'
\item 'male'
\end{enumerate*}


    
    \begin{Verbatim}[commandchars=\\\{\}]
{\color{incolor}In [{\color{incolor}40}]:} \PY{k+kp}{summary}\PY{p}{(}x\PY{p}{)}
\end{Verbatim}


    \begin{description*}
\item[fmale] 2
\item[male] 4
\end{description*}


    
    \begin{Verbatim}[commandchars=\\\{\}]
{\color{incolor}In [{\color{incolor}41}]:} \PY{c+c1}{\PYZsh{}change the order of levels}
         \PY{c+c1}{\PYZsh{}this is important in linear regression. The first level is used as the baseline level.}
         x \PY{o}{\PYZlt{}\PYZhy{}} \PY{k+kp}{factor}\PY{p}{(}\PY{k+kt}{c}\PY{p}{(}\PY{l+s}{\PYZdq{}}\PY{l+s}{male\PYZdq{}}\PY{p}{,} \PY{l+s}{\PYZdq{}}\PY{l+s}{fmale\PYZdq{}}\PY{p}{,} \PY{l+s}{\PYZdq{}}\PY{l+s}{male\PYZdq{}}\PY{p}{,} \PY{l+s}{\PYZdq{}}\PY{l+s}{male\PYZdq{}}\PY{p}{,} \PY{l+s}{\PYZdq{}}\PY{l+s}{fmale\PYZdq{}}\PY{p}{,} \PY{l+s}{\PYZdq{}}\PY{l+s}{male\PYZdq{}}\PY{p}{)}\PY{p}{,} levels\PY{o}{=}\PY{k+kt}{c}\PY{p}{(}\PY{l+s}{\PYZdq{}}\PY{l+s}{male\PYZdq{}}\PY{p}{,} \PY{l+s}{\PYZdq{}}\PY{l+s}{fmale\PYZdq{}}\PY{p}{)}\PY{p}{)}
         \PY{k+kp}{print}\PY{p}{(}x\PY{p}{)}
\end{Verbatim}


    \begin{Verbatim}[commandchars=\\\{\}]
[1] male  fmale male  male  fmale male 
Levels: male fmale

    \end{Verbatim}

    \begin{Verbatim}[commandchars=\\\{\}]
{\color{incolor}In [{\color{incolor}42}]:} d \PY{o}{\PYZlt{}\PYZhy{}} \PY{k+kt}{c}\PY{p}{(}\PY{l+m}{1}\PY{p}{,}\PY{l+m}{1}\PY{p}{,}\PY{l+m}{2}\PY{p}{,}\PY{l+m}{3}\PY{p}{,}\PY{l+m}{1}\PY{p}{,}\PY{l+m}{3}\PY{p}{,}\PY{l+m}{3}\PY{p}{,}\PY{l+m}{2}\PY{p}{)}
         d\PY{p}{[}\PY{l+m}{1}\PY{p}{]}\PY{o}{+}d\PY{p}{[}\PY{l+m}{2}\PY{p}{]} \PY{c+c1}{\PYZsh{} integers}
         fd \PY{o}{\PYZlt{}\PYZhy{}} \PY{k+kp}{factor}\PY{p}{(}d\PY{p}{)}
         \PY{k+kp}{print}\PY{p}{(}fd\PY{p}{)}
         fd\PY{p}{[}\PY{l+m}{1}\PY{p}{]}\PY{o}{+}fd\PY{p}{[}\PY{l+m}{2}\PY{p}{]} \PY{c+c1}{\PYZsh{}factors, you will get warning}
         \PY{k+kp}{unclass}\PY{p}{(}fd\PY{p}{)} \PY{c+c1}{\PYZsh{} bring down to integer vector}
\end{Verbatim}


    2

    
    \begin{Verbatim}[commandchars=\\\{\}]
[1] 1 1 2 3 1 3 3 2
Levels: 1 2 3

    \end{Verbatim}

    \begin{Verbatim}[commandchars=\\\{\}]
Warning message:
In Ops.factor(fd[1], fd[2]): ‘+’ not meaningful for factors
    \end{Verbatim}

    
    \begin{verbatim}
[1] NA
    \end{verbatim}

    
    \begin{enumerate*}
\item 1
\item 1
\item 2
\item 3
\item 1
\item 3
\item 3
\item 2
\end{enumerate*}


    
    \begin{Verbatim}[commandchars=\\\{\}]
{\color{incolor}In [{\color{incolor}43}]:} rd \PY{o}{\PYZlt{}\PYZhy{}} \PY{k+kp}{factor}\PY{p}{(}d\PY{p}{,} labels\PY{o}{=}\PY{k+kt}{c}\PY{p}{(}\PY{l+s}{\PYZdq{}}\PY{l+s}{A\PYZdq{}}\PY{p}{,} \PY{l+s}{\PYZdq{}}\PY{l+s}{B\PYZdq{}}\PY{p}{,} \PY{l+s}{\PYZdq{}}\PY{l+s}{C\PYZdq{}}\PY{p}{)}\PY{p}{)} \PY{c+c1}{\PYZsh{} factor is as an integer vector where each integer has a label}
         \PY{k+kp}{print}\PY{p}{(}rd\PY{p}{)}
\end{Verbatim}


    \begin{Verbatim}[commandchars=\\\{\}]
[1] A A B C A C C B
Levels: A B C

    \end{Verbatim}

    \begin{Verbatim}[commandchars=\\\{\}]
{\color{incolor}In [{\color{incolor}44}]:} \PY{k+kp}{levels}\PY{p}{(}rd\PY{p}{)} \PY{o}{\PYZlt{}\PYZhy{}} \PY{k+kt}{c}\PY{p}{(}\PY{l+s}{\PYZdq{}}\PY{l+s}{AA\PYZdq{}}\PY{p}{,} \PY{l+s}{\PYZdq{}}\PY{l+s}{BB\PYZdq{}}\PY{p}{,} \PY{l+s}{\PYZdq{}}\PY{l+s}{CC\PYZdq{}}\PY{p}{)}
         \PY{k+kp}{print}\PY{p}{(}rd\PY{p}{)}
\end{Verbatim}


    \begin{Verbatim}[commandchars=\\\{\}]
[1] AA AA BB CC AA CC CC BB
Levels: AA BB CC

    \end{Verbatim}

    \begin{Verbatim}[commandchars=\\\{\}]
{\color{incolor}In [{\color{incolor}45}]:} \PY{k+kp}{is.factor}\PY{p}{(}d\PY{p}{)}
         \PY{k+kp}{is.factor}\PY{p}{(}fd\PY{p}{)}
\end{Verbatim}


    FALSE

    
    TRUE

    
    \begin{Verbatim}[commandchars=\\\{\}]
{\color{incolor}In [{\color{incolor}9}]:} \PY{c+c1}{\PYZsh{}ordered factor variable}
        x1 \PY{o}{\PYZlt{}\PYZhy{}} \PY{k+kp}{factor}\PY{p}{(}\PY{k+kt}{c}\PY{p}{(}\PY{l+s}{\PYZdq{}}\PY{l+s}{low\PYZdq{}}\PY{p}{,} \PY{l+s}{\PYZdq{}}\PY{l+s}{high\PYZdq{}}\PY{p}{,} \PY{l+s}{\PYZdq{}}\PY{l+s}{medium\PYZdq{}}\PY{p}{,} \PY{l+s}{\PYZdq{}}\PY{l+s}{high\PYZdq{}}\PY{p}{,} \PY{l+s}{\PYZdq{}}\PY{l+s}{low\PYZdq{}}\PY{p}{,} \PY{l+s}{\PYZdq{}}\PY{l+s}{medium\PYZdq{}}\PY{p}{,} \PY{l+s}{\PYZdq{}}\PY{l+s}{high\PYZdq{}}\PY{p}{)}\PY{p}{)}
        \PY{k+kp}{print}\PY{p}{(}x1\PY{p}{)}
        x1f \PY{o}{\PYZlt{}\PYZhy{}} \PY{k+kp}{factor}\PY{p}{(}x1\PY{p}{,} levels \PY{o}{=} \PY{k+kt}{c}\PY{p}{(}\PY{l+s}{\PYZdq{}}\PY{l+s}{low\PYZdq{}}\PY{p}{,} \PY{l+s}{\PYZdq{}}\PY{l+s}{medium\PYZdq{}}\PY{p}{,} \PY{l+s}{\PYZdq{}}\PY{l+s}{high\PYZdq{}}\PY{p}{)}\PY{p}{)}
        \PY{k+kp}{print}\PY{p}{(}x1f\PY{p}{)}
\end{Verbatim}


    \begin{Verbatim}[commandchars=\\\{\}]
[1] low    high   medium high   low    medium high  
Levels: high low medium
[1] low    high   medium high   low    medium high  
Levels: low medium high

    \end{Verbatim}

    \begin{Verbatim}[commandchars=\\\{\}]
{\color{incolor}In [{\color{incolor}19}]:} x1o \PY{o}{\PYZlt{}\PYZhy{}} \PY{k+kp}{ordered}\PY{p}{(}x1\PY{p}{,} levels \PY{o}{=} \PY{k+kt}{c}\PY{p}{(}\PY{l+s}{\PYZdq{}}\PY{l+s}{low\PYZdq{}}\PY{p}{,} \PY{l+s}{\PYZdq{}}\PY{l+s}{medium\PYZdq{}}\PY{p}{,} \PY{l+s}{\PYZdq{}}\PY{l+s}{high\PYZdq{}}\PY{p}{)}\PY{p}{)}
         \PY{k+kp}{print}\PY{p}{(}x1o\PY{p}{)}
\end{Verbatim}


    \begin{Verbatim}[commandchars=\\\{\}]
[1] low    high   medium high   low    medium high  
Levels: low < medium < high

    \end{Verbatim}

    \begin{Verbatim}[commandchars=\\\{\}]
{\color{incolor}In [{\color{incolor}21}]:} \PY{k+kp}{min}\PY{p}{(}x1o\PY{p}{)} \PY{c+c1}{\PYZsh{}\PYZsh{} works!}
\end{Verbatim}


    low

    
    \begin{Verbatim}[commandchars=\\\{\}]
{\color{incolor}In [{\color{incolor}16}]:} \PY{k+kp}{is.factor}\PY{p}{(}x1o\PY{p}{)}
\end{Verbatim}


    TRUE

    
    \begin{Verbatim}[commandchars=\\\{\}]
{\color{incolor}In [{\color{incolor}17}]:} \PY{k+kp}{attributes}\PY{p}{(}x1o\PY{p}{)}
\end{Verbatim}


    \begin{description}
\item[\$levels] \begin{enumerate*}
\item 'low'
\item 'medium'
\item 'high'
\end{enumerate*}

\item[\$class] \begin{enumerate*}
\item 'ordered'
\item 'factor'
\end{enumerate*}

\end{description}


    
    By using the gl() function, we can generate factor levels . It takes two
integers as input which indicates how many levels and how many times
each level. * gl(n, m, labels) * n is the number of levels * m is the
number of repeatitions * labels is a vector of labels

    \begin{Verbatim}[commandchars=\\\{\}]
{\color{incolor}In [{\color{incolor}22}]:} v \PY{o}{\PYZlt{}\PYZhy{}} \PY{k+kp}{gl}\PY{p}{(}\PY{l+m}{3}\PY{p}{,} \PY{l+m}{4}\PY{p}{,} labels \PY{o}{=} \PY{k+kt}{c}\PY{p}{(}\PY{l+s}{\PYZdq{}}\PY{l+s}{H1\PYZdq{}}\PY{p}{,} \PY{l+s}{\PYZdq{}}\PY{l+s}{H2\PYZdq{}}\PY{p}{,}\PY{l+s}{\PYZdq{}}\PY{l+s}{H3\PYZdq{}}\PY{p}{)}\PY{p}{)}
         \PY{k+kp}{print}\PY{p}{(}v\PY{p}{)}
\end{Verbatim}


    \begin{Verbatim}[commandchars=\\\{\}]
 [1] H1 H1 H1 H1 H2 H2 H2 H2 H3 H3 H3 H3
Levels: H1 H2 H3

    \end{Verbatim}

    \begin{Verbatim}[commandchars=\\\{\}]
{\color{incolor}In [{\color{incolor}23}]:} \PY{k+kp}{class}\PY{p}{(}v\PY{p}{)}
\end{Verbatim}


    'factor'

    
    \subsection{2.7 Missing Values}\label{missing-values}

A variable might not have a value, ot its value might missing. In R
missing values are displayed by the symbol NA (not avaiable). * NA, not
available * Makes certain calculations impossible * is.na() * is.nan() *
NA values have class

    \begin{Verbatim}[commandchars=\\\{\}]
{\color{incolor}In [{\color{incolor}13}]:} x1 \PY{o}{\PYZlt{}\PYZhy{}} \PY{k+kt}{c}\PY{p}{(}\PY{l+m}{4}\PY{p}{,} \PY{l+m}{2.5}\PY{p}{,} \PY{l+m}{3}\PY{p}{,} \PY{k+kc}{NA}\PY{p}{,} \PY{l+m}{1}\PY{p}{)}
         \PY{k+kp}{summary}\PY{p}{(}x1\PY{p}{)}  \PY{c+c1}{\PYZsh{} Works with NA}
         \PY{k+kp}{mean}\PY{p}{(}x1\PY{p}{)}  \PY{c+c1}{\PYZsh{} Doesn\PYZsq{}t work}
         \PY{k+kp}{mean}\PY{p}{(}x1\PY{p}{,} na.rm\PY{o}{=}\PY{k+kc}{TRUE}\PY{p}{)}
\end{Verbatim}


    
    \begin{verbatim}
   Min. 1st Qu.  Median    Mean 3rd Qu.    Max.    NA's 
  1.000   2.125   2.750   2.625   3.250   4.000       1 
    \end{verbatim}

    
    
    \begin{verbatim}
[1] NA
    \end{verbatim}

    
    2.625

    
    \begin{Verbatim}[commandchars=\\\{\}]
{\color{incolor}In [{\color{incolor}25}]:} \PY{k+kp}{is.na}\PY{p}{(}x1\PY{p}{)}
\end{Verbatim}


    \begin{enumerate*}
\item FALSE
\item FALSE
\item FALSE
\item TRUE
\item FALSE
\end{enumerate*}


    
    \begin{Verbatim}[commandchars=\\\{\}]
{\color{incolor}In [{\color{incolor}26}]:} \PY{c+c1}{\PYZsh{} To find missing values}
         \PY{k+kp}{which}\PY{p}{(}\PY{k+kp}{is.na}\PY{p}{(}x1\PY{p}{)}\PY{p}{)}  \PY{c+c1}{\PYZsh{} Give index number}
\end{Verbatim}


    4

    
    \begin{Verbatim}[commandchars=\\\{\}]
{\color{incolor}In [{\color{incolor}29}]:} \PY{c+c1}{\PYZsh{} Ignore missing values with na.rm = T}
         \PY{k+kp}{mean}\PY{p}{(}x1\PY{p}{,} na.rm \PY{o}{=} \PY{n+nb+bp}{T}\PY{p}{)}
\end{Verbatim}


    2.625

    
    \begin{Verbatim}[commandchars=\\\{\}]
{\color{incolor}In [{\color{incolor}30}]:} \PY{c+c1}{\PYZsh{} Replace missing values with 0 (or other number)}
         \PY{c+c1}{\PYZsh{} In data wrangling you will learn a lot about this.}
         x2 \PY{o}{\PYZlt{}\PYZhy{}} x1
         x2\PY{p}{[}\PY{k+kp}{is.na}\PY{p}{(}x2\PY{p}{)}\PY{p}{]} \PY{o}{\PYZlt{}\PYZhy{}} \PY{l+m}{0}
         x2
\end{Verbatim}


    \begin{enumerate*}
\item 4
\item 2.5
\item 3
\item 0
\item 1
\end{enumerate*}


    
    \subsection{2.8 Subsetting}\label{subsetting}

\begin{itemize}
\tightlist
\item
  {[}{]} always returns an object of the same class
\item
  {[}{[}{]}{]} is used to extract elements from a list fo dataframe. It
  always return a single element.
\item
  \(\$\) to extract elements from a list or dataframe unsing a name
\end{itemize}

    \begin{Verbatim}[commandchars=\\\{\}]
{\color{incolor}In [{\color{incolor}51}]:} x \PY{o}{\PYZlt{}\PYZhy{}} \PY{k+kt}{c}\PY{p}{(}\PY{l+s}{\PYZdq{}}\PY{l+s}{a1\PYZdq{}}\PY{p}{,} \PY{l+s}{\PYZdq{}}\PY{l+s}{a2\PYZdq{}}\PY{p}{,} \PY{l+s}{\PYZdq{}}\PY{l+s}{a3\PYZdq{}}\PY{p}{,} \PY{l+s}{\PYZdq{}}\PY{l+s}{a4\PYZdq{}}\PY{p}{,} \PY{l+s}{\PYZdq{}}\PY{l+s}{a5\PYZdq{}}\PY{p}{,} \PY{l+s}{\PYZdq{}}\PY{l+s}{a6\PYZdq{}}\PY{p}{)}
\end{Verbatim}


    \begin{Verbatim}[commandchars=\\\{\}]
{\color{incolor}In [{\color{incolor}52}]:} x\PY{p}{[}\PY{l+m}{1}\PY{p}{]} \PY{c+c1}{\PYZsh{}extracts the first item. it\PYZsq{}s a vector}
         x\PY{p}{[}\PY{l+m}{2}\PY{o}{:}\PY{l+m}{5}\PY{p}{]} \PY{c+c1}{\PYZsh{} extracts a sequence. it\PYZsq{}s a vector}
\end{Verbatim}


    'a1'

    
    \begin{enumerate*}
\item 'a2'
\item 'a3'
\item 'a4'
\item 'a5'
\end{enumerate*}


    
    \begin{Verbatim}[commandchars=\\\{\}]
{\color{incolor}In [{\color{incolor}35}]:} x \PY{o}{\PYZlt{}\PYZhy{}} \PY{k+kt}{list}\PY{p}{(}prime\PY{o}{=}\PY{k+kt}{c}\PY{p}{(}\PY{l+m}{2}\PY{p}{,}\PY{l+m}{3}\PY{p}{,}\PY{l+m}{5}\PY{p}{,}\PY{l+m}{7}\PY{p}{)}\PY{p}{,} even\PY{o}{=}\PY{k+kt}{c}\PY{p}{(}\PY{l+m}{0}\PY{p}{,}\PY{l+m}{2}\PY{p}{,}\PY{l+m}{4}\PY{p}{,}\PY{l+m}{6}\PY{p}{)}\PY{p}{,} odd\PY{o}{=}\PY{k+kt}{c}\PY{p}{(}\PY{l+m}{1}\PY{p}{,}\PY{l+m}{3}\PY{p}{,}\PY{l+m}{5}\PY{p}{,}\PY{l+m}{7}\PY{p}{)}\PY{p}{,} digit\PY{o}{=}\PY{l+m}{3.14}\PY{p}{)}
\end{Verbatim}


    \begin{Verbatim}[commandchars=\\\{\}]
{\color{incolor}In [{\color{incolor}36}]:} \PY{k+kp}{print}\PY{p}{(}x\PY{p}{)}
\end{Verbatim}


    \begin{Verbatim}[commandchars=\\\{\}]
\$prime
[1] 2 3 5 7

\$even
[1] 0 2 4 6

\$odd
[1] 1 3 5 7

\$digit
[1] 3.14


    \end{Verbatim}

    \begin{Verbatim}[commandchars=\\\{\}]
{\color{incolor}In [{\color{incolor}40}]:} \PY{k+kp}{print}\PY{p}{(}x\PY{p}{[}\PY{l+m}{1}\PY{p}{]}\PY{p}{)} \PY{c+c1}{\PYZsh{}extracts the first element of the list, and it is a list}
         \PY{k+kp}{class}\PY{p}{(}x\PY{p}{[}\PY{l+m}{1}\PY{p}{]}\PY{p}{)}
\end{Verbatim}


    \begin{Verbatim}[commandchars=\\\{\}]
\$prime
[1] 2 3 5 7


    \end{Verbatim}

    'list'

    
    \begin{Verbatim}[commandchars=\\\{\}]
{\color{incolor}In [{\color{incolor}44}]:} \PY{k+kp}{print}\PY{p}{(}x\PY{p}{[[}\PY{l+m}{1}\PY{p}{]]}\PY{p}{)} \PY{c+c1}{\PYZsh{}extracts the first element and returns a vector.}
\end{Verbatim}


    \begin{Verbatim}[commandchars=\\\{\}]
[1] 2 3 5 7

    \end{Verbatim}

    \begin{Verbatim}[commandchars=\\\{\}]
{\color{incolor}In [{\color{incolor}57}]:} \PY{k+kp}{print}\PY{p}{(}x\PY{p}{[}\PY{l+m}{4}\PY{p}{]}\PY{p}{)}
\end{Verbatim}


    \begin{Verbatim}[commandchars=\\\{\}]
\$digit
[1] 3.14


    \end{Verbatim}

    \begin{Verbatim}[commandchars=\\\{\}]
{\color{incolor}In [{\color{incolor}58}]:} \PY{k+kp}{print}\PY{p}{(}x\PY{p}{[[}\PY{l+m}{4}\PY{p}{]]}\PY{p}{)}
\end{Verbatim}


    \begin{Verbatim}[commandchars=\\\{\}]
[1] 3.14

    \end{Verbatim}

    \begin{Verbatim}[commandchars=\\\{\}]
{\color{incolor}In [{\color{incolor}59}]:} x\PY{o}{\PYZdl{}}digit
\end{Verbatim}


    3.14

    
    \begin{Verbatim}[commandchars=\\\{\}]
{\color{incolor}In [{\color{incolor}60}]:} x\PY{p}{[}\PY{k+kt}{c}\PY{p}{(}\PY{l+m}{1}\PY{p}{,}\PY{l+m}{4}\PY{p}{)}\PY{p}{]}
\end{Verbatim}


    \begin{description}
\item[\$prime] \begin{enumerate*}
\item 2
\item 3
\item 5
\item 7
\end{enumerate*}

\item[\$digit] 3.14
\end{description}


    
    \subsection{2.9 Vectorised Operations}\label{vectorised-operations}

Makes life much easier!! We can treat vectors as single variables in R.
sometimes we want to apply a particular calculation on all the members
of a vector, or between two vectors.

    \begin{Verbatim}[commandchars=\\\{\}]
{\color{incolor}In [{\color{incolor}47}]:} x \PY{o}{\PYZlt{}\PYZhy{}} \PY{l+m}{1}\PY{o}{:}\PY{l+m}{4}
         \PY{l+m}{2}\PY{o}{*}x
         y \PY{o}{\PYZlt{}\PYZhy{}} \PY{l+m}{2}\PY{o}{:}\PY{l+m}{5}
         \PY{k+kp}{print}\PY{p}{(}x\PY{o}{+}y\PY{p}{)}
\end{Verbatim}


    \begin{enumerate*}
\item 2
\item 4
\item 6
\item 8
\end{enumerate*}


    
    \begin{Verbatim}[commandchars=\\\{\}]
[1] 3 5 7 9

    \end{Verbatim}

    \begin{Verbatim}[commandchars=\\\{\}]
{\color{incolor}In [{\color{incolor}48}]:} x\PY{p}{[}x\PY{o}{\PYZgt{}}\PY{l+m}{2}\PY{p}{]}
\end{Verbatim}


    \begin{enumerate*}
\item 3
\item 4
\end{enumerate*}


    
    \begin{Verbatim}[commandchars=\\\{\}]
{\color{incolor}In [{\color{incolor}62}]:} \PY{k+kp}{print}\PY{p}{(}x\PY{o}{*}y\PY{p}{)}
\end{Verbatim}


    \begin{Verbatim}[commandchars=\\\{\}]
[1]  2  6 12 20

    \end{Verbatim}

    \begin{Verbatim}[commandchars=\\\{\}]
{\color{incolor}In [{\color{incolor}63}]:} \PY{k+kp}{print}\PY{p}{(}x\PY{o}{\PYZgt{}}y\PY{p}{)}
\end{Verbatim}


    \begin{Verbatim}[commandchars=\\\{\}]
[1] FALSE FALSE FALSE FALSE

    \end{Verbatim}

    \begin{Verbatim}[commandchars=\\\{\}]
{\color{incolor}In [{\color{incolor}64}]:} \PY{c+c1}{\PYZsh{} Matrices will be covered soon.}
         m1 \PY{o}{\PYZlt{}\PYZhy{}} \PY{k+kt}{matrix}\PY{p}{(}\PY{l+m}{1}\PY{o}{:}\PY{l+m}{4}\PY{p}{,}\PY{l+m}{2}\PY{p}{,}\PY{l+m}{2}\PY{p}{)}
         m2 \PY{o}{\PYZlt{}\PYZhy{}} \PY{k+kt}{matrix}\PY{p}{(}\PY{l+m}{2}\PY{o}{:}\PY{l+m}{5}\PY{p}{,} \PY{l+m}{2}\PY{p}{,}\PY{l+m}{2}\PY{p}{)}
\end{Verbatim}


    \begin{Verbatim}[commandchars=\\\{\}]
{\color{incolor}In [{\color{incolor}65}]:} m1\PY{o}{+}m2
         m1\PY{o}{*}m2
         m1\PY{o}{\PYZpc{}*\PYZpc{}}m2 \PY{c+c1}{\PYZsh{}matrix multiplicatin}
\end{Verbatim}


    \begin{tabular}{ll}
	 3 & 7\\
	 5 & 9\\
\end{tabular}


    
    \begin{tabular}{ll}
	 2  & 12\\
	 6  & 20\\
\end{tabular}


    
    \begin{tabular}{ll}
	 11 & 19\\
	 16 & 28\\
\end{tabular}


    
    R can perform functions over entire vectors and can be used to select
certain elements within a vector. Here is a alist of more frequent
functions: * max(x)\\
* min(x) * sum(x) * mean(x) * var(x) * sd(x) * median(x) * range(x)


    % Add a bibliography block to the postdoc
    
    
    
    \end{document}
